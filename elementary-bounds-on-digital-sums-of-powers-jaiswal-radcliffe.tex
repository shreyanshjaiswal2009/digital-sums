\documentclass[12pt]{amsart} 
\usepackage{geometry}
\geometry{letterpaper}
\usepackage{amsmath}
\usepackage{amssymb}
\usepackage{graphicx}
\usepackage{hyperref}
\usepackage{xurl}
\usepackage{parskip}
\usepackage{xlop}
\usepackage{mathtools}
\newtheorem{theorem}{Theorem}
\newtheorem{corollary}{Corollary}
\newtheorem{lemma}{Lemma}
\DeclareMathOperator{\lcm}{lcm}
\setlength{\tabcolsep}{2pt}

\title{Elementary Bounds on Digital Sums of Powers, Factorials, and LCMs}
\author{Shreyansh Jaiswal and David G. Radcliffe}
\date{December 10, 2025}

\begin{document}

\begin{abstract}
	We prove lower bounds on digital sums of powers, multiples of powers, factorials,
	and the least common multiple of $\{1,\ldots, n\}$, using only elementary number theory.
\end{abstract}

\maketitle

\section{Introduction}

This expository article establishes lower bounds on digital sums
of powers, multiples of powers, factorials, and the least common
multiple of $\{1,\ldots, n\}$, using only elementary number theory.

We were inspired by the following problem, which was posed and solved
by Wac{\l}aw Sierpi{\'n}ski~\cite[Problem 209]{sierpinski1970}:
\begin{quote}
	\emph{Prove that the sum of digits of the number $2^{n}$ (in decimal system)
		increases to infinity with $n$.}
\end{quote}

The reader is urged to attempt this problem on their own before proceeding.
Note that it is not enough to prove that the sum of digits of $2^{n}$ is
unbounded, since the sequence is not monotonic.

Consider the sequence of powers of $2$ (sequence \href{https://oeis.org/A000079}{A000079}
in the OEIS):

\[
	1, 2, 4, 8, 16, 32, 64, 128, 256, 512, 1024, \ldots .
\]

This sequence grows very rapidly. Now, let us define another sequence, by adding
the decimal digits of each power of $2$. For example, $16$ becomes $1+6=7$, and $32$
becomes $3+2=5$. The first few terms of this new sequence
(\href{https://oeis.org/A001370}{A001370}) are listed below.

\[
	1, 2, 4, 8, 7, 5, 10, 11, 13, 8, 7, \ldots .
\]

This new sequence grows much more slowly, and it is not monotonic.
Nevertheless, it is reasonable to conjecture that it tends to infinity.
Indeed, one might guess that the sum of the decimal digits of $2^{n}$
is approximately equal to $4.5 n \log_{10}2$,
since $2^{n}$ has $\lfloor n \log_{10}2\rfloor + 1$ decimal digits,
and the digits seem to be approximately uniformly distributed among
$0, 1, 2, \ldots, 9$.
However, this stronger conjecture remains to be proved.
See Figure~\ref{fig:s10_scatter}.

\begin{figure}
	\centering
	\includegraphics[width=0.8\linewidth]{s10_scatter.png}
	\caption{Scatter plot of the digital sum of $2^n$ for $n \le 100$
		together with the conjectured linear approximation.}
	\label{fig:s10_scatter}
\end{figure}

We prove in Section~\ref{sec:powers-of-two} that the digital
sum of $2^n$ is greater than $\log_4 n$ for all $n \ge 1$.
But first, we introduce some notation.

\section{Notation and terminology}

For integers $n \ge 0$ and $b \ge 2$, the \emph{base-$b$ expansion} of $n$
is the unique representation of the form
\[
	n = \sum_{i=0}^{\infty} d_i b^i, \qquad d_i \in \{0, 1, \ldots, b-1\}.
\]
The integers $d_i$ are the \emph{base-$b$} digits of $n$; all but finitely
many of these digits are zero.

For an integer $b \ge 2$, we write $s_{b}(n)$ for the sum of the base-$b$
digits of $n$, and $c_{b}(n)$ for the number of nonzero digits in that expansion.
These functions are equivalent up to a constant factor,
since $c_{b}(n) \le s_{b}(n) \le (b-1) c_{b}(n)$ for all $n$ and $b$;
so we focus on $c_{b}(n)$.

The function $s_b$ is \emph{subadditive}: $s_b(m+n) \le s_b(m) + s_b(n)$
for all positive integers $m$ and $n$. Equality
holds if no carries occur in the digitwise addition of $m$ and $n$.
Otherwise, each carry reduces the digital sum by $b-1$.
The function $c_b$ is likewise subadditive.

For a prime $p$, $\nu_{p}(n)$ denotes the exponent of $p$ in the
prime factorization of $n$. If $p$ does not divide $n$ then $\nu_{p}(n) = 0$.
The function $\nu_p$ is \emph{completely additive}:
$\nu_p(mn) = \nu_p(m) + \nu_p(n)$ for all positive integers $m$ and $n$.

Recall that the \emph{floor} of $x$, denoted $\lfloor x \rfloor$,
is the greatest integer $n$ such that $n \le x$.
The \emph{ceiling} of $x$, denoted
$\lceil x \rceil$, is the least integer $n$ such that $n \ge x$.

We use Bachmann-Landau-Knuth notations~\cite{knuth1997} to describe the approximate size of functions.
Let $f$ and $g$ be real-valued functions defined on a domain $D$,
usually the set of positive integers.

One writes
\[
	f(n) = O(g(n))
\]
if there exists a positive real number $C$ such that
\[
	|f(n)| \le C g(n) \qquad \text{for all } n \in D.
\]

The notation
\[
	f(n) = o(g(n))
\]
means that
\[
	\lim_{n\to\infty} \frac{f(n)}{g(n)} = 0.
\]

In particular, $O(1)$ denotes a bounded function, and $o(1)$ denotes
a function that tends to 0 as $n \to \infty$.

Finally,
\[
	f(n) \asymp g(n)
\]
means that there exist positive real numbers $C$ and $C'$ such that
\[
	C g(n) < |f(n)| < C' g(n) \qquad \text{for all } n \in D.
\]

\section{Digital sums of powers of two}
\label{sec:powers-of-two}

We present an informal proof that $c_{10}(2^{n})$ tends to infinity as $n \to \infty$.
See \cite{radcliffe2016} for an alternative approach.

Let $n$ be a positive integer, and write the decimal expansion of $2^n$ as
$2^n = \sum_{i=0}^{\infty} d_i 10^i$, where $d_i \in \{0, \ldots, 9\}$
and $d_i = 0$ for all but finitely many terms.
Note that the final digit $d_0$ of $2^{n}$ cannot be zero,
since $2^n$ is not divisible by $10$.

If $2^{n} > 10$, then $2^{n}$ is divisible by $16$; so
$2^{n} \bmod 10^{4}$, the number formed by the last four digits of $2^{n}$, is
also divisible by $16$. If the first three of these digits were zero, then $2^{n}
	\bmod 10^{4}$ would be less than $10$, which is impossible.
So at least one of the digits $d_1, d_2, d_3$ is nonzero.

If $2^{n} > 10^4$, then $2^{n}$ is divisible by $2^{14}$; so
$2^{n} \bmod 10^{14}$, the number formed by the last $14$ digits of $2^{n}$,
is also divisible by $2^{14}$. If the first $10$ of these digits were zero,
then $2^{n} \bmod 10^{14}$ would be less than $10^{4}$, which is impossible.
So at least one of the digits $d_4, d_5, \ldots, d_{13}$ is nonzero.

We can continue in this way, finding longer and longer non-overlapping blocks
of digits, each containing at least one nonzero digit, as illustrated in 
Figure~\ref{fig-blocks}.
This proves that $c_{10}(2^n)$ tends to infinity as $n \to \infty$.

\begin{figure}
	\begin{tabular}{lcrrrrr}
		$2^0$     & = &   &                                   &            &     & 1 \\
		$2^4$     & = &   &                                   &            & 1   & 6 \\
		$2^{14}$  & = &   &                                   & 1          & 638 & 4 \\
		$2^{47}$  & = &   & 1                                 & 4073748835 & 532 & 8 \\
		$2^{157}$ & = & 1 & 826877046663628647754606040895353 & 7745699156 & 787 & 2
	\end{tabular}
	\caption{Digits of $2^n$ subdivided into blocks.
		Each block contains at least one nonzero digit.}
	\label{fig-blocks}
\end{figure}

Let us formalize this argument.

\begin{theorem}
	Let $(e(k))_{k\ge1}$ be a sequence of integers such that
	$e(1) \ge 1$ and $2^{e(k)} > 10^{e(k-1)}$ for all $k \ge 2$.
	If $n$ is a positive integer that is divisible by $2^{e(k)}$
	but not divisible by $10$, then $c_{10}(n) \ge k$.
	\label{base-ten-lower-bound}
\end{theorem}

\begin{proof}
	We argue by induction on $k$.
	The case $k=1$ is immediate, since any positive integer
	has at least one nonzero digit.

	Assume now that $k \ge 2$, and that the statement holds for $k-1$.
	Let $n$ be a positive integer divisible by $2^{e(k)}$ but not by $10$.
	Apply the division algorithm to write
	\[
		n = 10^{e(k-1)} q + r, \qquad 0 \le r < 10^{e(k-1)},
	\]
	for integers $q$, $r$.

	Because $n \ge 2^{e(k)} > 10^{e(k-1)}$ by hypothesis,
	the quotient satisfies $q \ge 1$.

	Next, both $n$ and $10^{e(k-1)} q$ are divisible by $2^{e(k-1)}$,
	hence their difference
	\[
		r = n - 10^{e(k-1)} q
	\]
	is also divisible by $2^{e(k-1)}$.

	Moreover, $r$ is not divisible by $10$, since $n$ is not divisible by $10$.

	Therefore, $c_{10}(r) \ge k - 1$ by the induction hypothesis.

	Finally, the decimal expansion of $n$ is obtained by concatenating
	the decimal expansion of $q$ with the (possibly zero-padded)
	expansion of $r$. Thus,
	\[
		c_{10}(n) = c_{10}(q) + c_{10}(r) \ge 1 + (k - 1) = k.
	\]
	This completes the proof.

\end{proof}

\begin{corollary}
	Let $a$ be a positive integer that is divisible by $2$ but not divisible by $10$.
	Then $c_{10}(a^n) \ge \log_4(n)$ for all $n \ge 1$.
	In particular,
	$\lim_{n\to \infty}c_{10}(a^{n}) = \infty$.
	\label{even-powers-base-ten-limit}
\end{corollary}

\begin{proof}
	Let $e(k) = 4^{k-1}$ for $k \ge 1$.
	This sequence satisfies $e(1) \ge 1$
	and
	\[
		2^{e(k)} > 10^{e(k-1)}
	\]
	for all $k \ge 2$.

	Let $n\ge 4$ be a positive integer, and let $k = \lceil \log_4 n \rceil$,
	so that $4^{k-1} < n \le 4^k$. Then $a^n$ is divisible by $2^n$,
	so $a^n$ is also divisible by $2^{e(k)}$.
	But $a^n$ is not divisible by $10$.

	Therefore,
	$c_{10}(a^n) \ge k \ge \log_4(n)$, by Theorem~\ref{base-ten-lower-bound}.
\end{proof}

\section{Generalizing to other bases}

Before extending the arguments of the previous section, we first situate
our results in the broader context of earlier work on digital sums of powers.
In 1973, Senge and Straus~\cite[Theorem 3]{senge-straus1973}
showed that for integers $a \ge 1$ and $b \ge 2$,
\[
	\lim_{n \to \infty} c_b(a^n) = \infty
	\quad\Longleftrightarrow\quad
	\frac{\log a}{\log b} \text{ is irrational}.
\]
Their result provides a qualitative criterion for unbounded digit growth,
but it does not yield any explicit lower bound.

Subsequently, Stewart~\cite[Theorem 2]{stewart1980} obtained the first
general quantitative estimate.  Under the same hypothesis of irrationality of
$\log(a)/\log(b)$, he proved that
\[
	c_b(a^n) > \frac{\log n}{\log\log n + C} - 1
\]
for all $n>4$, where $C$ depends only on $a$ and $b$.
This bound is extremely general but grows more slowly than logarithmic.

In this section we prove a logarithmic lower bound for $c_b(a^n)$ under
a set of hypotheses that are stronger than Stewart's but still hold in
many natural situations.  Our approach proceeds in two steps:
first we establish a structural digit-sum bound for integers that are 
divisible by powers of $a$ but not by $b$, and then we apply it to $a^n$
itself under an additional condition.
The first step (Theorem~\ref{thm:multiples-of-powers} below)
generalizes the corresponding base-$10$ argument from the previous section.

\begin{theorem}
	Let $a, b$ be integers such that $2 \le a < b$ and $a$ divides $b$.
	Let $(e(k))$ be a sequence of integers such that $e(1) \ge 1$ and
	$a^{e(k)} \ge b^{e(k-1)}$ for all $k \ge 2$.
	If $a^{e(k)}$ divides $n$ and $b \nmid n$, then $c_b(n) \ge k$.
	\label{thm:multiples-of-powers}
\end{theorem}

\begin{proof}
	Follow the proof of Theorem~\ref{base-ten-lower-bound}, but replace $2$
	with $a$ where needed, and replace $10$ with $b$.
\end{proof}

\begin{theorem}
	Let $a, b$ be integers such that $2 \le a < b$ and $a$ divides $b$.
	Suppose that $N$ is divisible by $a^n$ but not $b$. Then
	there exists $C > 0$, depending only on $a$ and $b$, such that
	\[
		c_b(N) > C \log n
	\]
	for $n$ sufficiently large.
	Indeed, we can choose any $0 < C < (\log(\log b / \log a))^{-1}$.
	\label{thm:cbn}
\end{theorem}

\begin{proof}

	Let $r = \log b\, / \log a$, and
	define $(e_k)$ by $e_1 = 1$ and $e_k = \lceil r e_{k-1} \rceil$
	for $k \ge 2$.
	It is routine to verify that $(e_k)$ satisfies the conditions of
	Theorem~\ref{thm:multiples-of-powers}.

	Then,
	\begin{align*}
		e_2 & < r + 1,             \\
		e_3 & < r^2 + r + 1,       \\
		e_4 & < r^3 + r^2 + r + 1, \\
	\end{align*}
	and for all $k \ge 1$,
	\begin{equation}
		\label{eq:ek-estimate}
		e_k < \sum_{i=0}^{k-1} r^{i}
		< \frac{r^k}{r - 1}.
	\end{equation}

	Now suppose that $n$ is an integer greater than $\displaystyle\frac{r}{r-1}$, and let
	\[
		k = \left\lfloor \frac{\log((r-1)n)}{\log r} \right\rfloor.
	\]
	Then \[
		1 \le k \le \frac{\log((r-1) n)}{\log r},
	\]
	which implies that
	\[
		n \ge \frac{r^k}{r-1}.
	\]
	Therefore $n > e_k$ by \eqref{eq:ek-estimate}, hence
	$c_b(N) \ge k$ by Theorem~\ref{thm:multiples-of-powers}.

	Choose $C$ so that $0 < C < (\log r)^{-1}$. Then
	\[
		\left\lfloor \frac{\log((r-1)n)}{\log r} \right\rfloor
		> C \log n
	\]
	for $n$ sufficiently large, which implies that $c_b(N) > C \log n$.
\end{proof}

The following lemma allows us to relax the condition that $b$ does not
divide $a^n$.

\begin{lemma}
	\label{lemma-1}
	Let $a, b \ge 2$ be integers such that $\log a\, / \log b$ is irrational.
	Suppose that $a^n = b^r u$.
	Then there exists a prime factor $p$ of $a$, and $C > 0$ depending
	only on $a$ and $b$, such that $\nu_p(u) \ge Cn$.
\end{lemma}

\begin{proof}
	Since $\log a\, / \log b$ is irrational, there exist primes $p$ and $q$
	such that
	\begin{equation}
		\nu_p(a) \nu_q(b) - \nu_q(a) \nu_p(b) > 0.
		\label{eq:nudiff}
	\end{equation}
	Otherwise, $\log a\, / \log b$ would be equal to the common ratio
	$\nu_p(a)\, / \nu_p(b)$ for any prime divisor $p$ of $b$.

	By comparing the valuations of $a^n$ with respect to $p$ and $q$,
	we obtain 
	\begin{equation}
		n \nu_p(a) = r \nu_p(b) + \nu_p(u)
		\label{eq:nup}
	\end{equation}
	and
	\begin{equation}
		n \nu_q(a) \ge r \nu_q(b).
		\label{eq:nuq}
	\end{equation}

	Combining \eqref{eq:nup} and \eqref{eq:nuq} yields
	\begin{equation}
		\label{eq:nupq}
			\nu_p(u) \ge n \nu_p(a) - n \frac{\nu_q(a)}{\nu_q(b)} \nu_p(b) \\
			         = C n,
	\end{equation}
	where \[
		C = \frac{\nu_p(a) \nu_q(b) - \nu_q(a) \nu_p(b)}{\nu_q(b)}.
	\]
	Finally, $C > 0$ by \eqref{eq:nudiff}.
\end{proof}

We now come to the main result of this section.

\begin{theorem}
	Let $a \ge 2$ and $b \ge 2$ be integers. Let $d$ be the smallest factor of $a$
	such that $\gcd(a/d, b) = 1$, and suppose that $\log(d) / \log(b)$ is irrational.
	Then $c_{b}(a^{n}) > C \log n$ for all $n \ge 1$, where $C > 0$ depends
	only on $a$ and $b$.
	\label{general-digit-sum}
\end{theorem}

\begin{proof}
	Let $a^n = b^r u$, where $b \nmid u$.
	If $\log d\, / \log b$ is irrational, then $\log a\, / \log b$ is also irrational;
	so Lemma~\ref{lemma-1} implies that there exists a prime factor $p$ of $a$, and $C' > 0$ depending
	only on $a$ and $b$, such that $\nu_p(u) \ge C' n$.

	But $p$ also divides $d$, since $\gcd(p, a/d) = 1$.
	Therefore, $c_b(u) > C \log n$ for some $C > 0$, by Theorem~\ref{thm:cbn}.

	Finally, $c_b(u) = c_b(a^n)$, since $m$ and $a^n$ have
	the same digit expansion, apart from trailing zeros.
\end{proof}

\section{Digital sums of factorials and LCMs}

The results of the previous section showed how information about the
prime-power divisibility of an integer can force its base-$b$ digital sum
to grow at least logarithmically.  Although we now turn to a different
family of integers, namely $n!$ and the least common multiple
$\Lambda_n = \lcm(1,\dots,n)$, the same guiding principle remains:
multiplicative structure can impose strong, and often very elementary,
constraints on possible base-$b$ representations.

In contrast with the situation for $a^n$, where prime-power divisibility
played a central role, the key structural feature for factorials and LCMs
is that both $n!$ and $\Lambda_n$ are divisible by large integers of the
form $b^r - 1$.
This observation leads immediately to logarithmic lower bounds on their
digital sums, provided one has an effective way to bound the digital sum of
a multiple of $b^r - 1$.

The following lemma, originally proved by Stolarsky~\cite{stolarsky1980}
for base $2$ and extended to general bases by Balog and
Dartyge~\cite{balog2012}, supplies exactly such a tool.

\begin{lemma}
	Let $m, r \ge 1$ and $b \ge 2$ be integers.
	If $m$ is divisible by $b^r - 1$,
	then $s_b(m) \ge (b-1)r$.
\end{lemma}

\begin{proof}
	Write the base-$b$ expansion of $m$ as a concatenation of $r$-digit blocks,
	so that
	\[
		m = \sum_{i=0}^{k-1} B_i b^{ri}, \qquad 0 \le B_i < b^r, \qquad B_{k-1} \ge 1.
	\]

	Define the \emph{block-sum} operator $G$ by
	\[
		G(m) = \sum_{i=0}^{k-1} B_i.
	\]
	Observe that
	$G(m) \equiv m \pmod{b^r - 1}$, since $b^r \equiv 1 \pmod{b^r - 1}$.
	Also, $G(m) < m$ for $m \ge b^r$, and $G(m) = m$ for $0 \le m < b^r$.

	Iterate $G$ on $m$: define $m_0 = m$ and $m_{t+1} = G(m_t)$.
	By the observations above, $(m_t)$ is a sequence of
	positive multiples of $b^r - 1$, which is strictly decreasing
	while its terms exceed $b^r - 1$. Therefore, the sequence must
	eventually reach $b^r - 1$, which is the unique positive multiple of $b^r - 1$
	that is less than $b^r$.

	Since $s_b$ is subadditive,
	\[
		s_b(G(m)) \le \sum_{i=0}^{k-1} s_b(B_i) = s_b(m).
	\]

	Therefore,
	\[
		s_b(m) \ge s_b(b^r - 1) = (b-1)r.
	\]
\end{proof}

This lemma has an immediate consequence for factorials and LCMs.
If $n \ge b^r - 1$, then both $n!$ and $\Lambda_n$ are divisible by $b^r - 1$,
and hence
\[
	s_b(n!) \ge (b-1)r, \qquad s_b(\Lambda_n) \ge (b-1)r.
\]
Since one may choose $r = \lfloor \log_b (n+1) \rfloor$, this yields
lower bounds of the form
\[
	s_b(n!) > C \log n, \qquad s_b(\Lambda_n) > C \log n
\]
for some constant $C > 0$ depending only on $b$.

Stronger bounds are known.
In 2015, Sanna~\cite{sanna2015} used this lemma,
together with more advanced methods, to prove that
\[
	s_b(n!) > C \log n \log \log \log n
\]
for all integers $n > e^{e}$ and all $b \ge 2$, where $C$ depends only on $b$.
The same estimate holds for $s_b(\Lambda_n)$.
Our interest here is not to compete with the best known results,
but rather to show that simple divisibility arguments
already imply logarithmic growth.

We conjecture that $s_{b}(n!) \asymp n \log n$ and $s_b(\Lambda_n) \asymp n$,
but this remains to be proved. See Figure~\ref{fig:s10_fact_lcm_scatter}.

\begin{figure}
	\centering
	\includegraphics[width=0.8\linewidth]{lcm_fact_scatter.png}
	\caption{Scatter plot of the digital sums of $n!$ and $\Lambda_n$
		for $n \le 100$,
		together with their conjectured approximations.}
	\label{fig:s10_fact_lcm_scatter}
\end{figure}

\section{Digital sums of powers: the general case}
\label{general-case}

In this final section, we prove that the number of nonzero digits in the base-$b$
expansion of $a^n$ tends to infinity as $n \to \infty$, provided that $\log(b)/\log(a)$
is irrational. This result appears in earlier work of Senge and Straus \cite{senge-straus1973}
and Stewart \cite{stewart1980}, but we present an argument that we hope is more accessible.

The irrationality condition is necessary. Indeed, if $\log(a)/\log(b) = r/s \in \mathbb{Q}$, then
\[
	a^{ns} = b^{nr}
\]
for every integer $n$, so $a^{ns}$ has only one nonzero digit in base $b$.

We may assume without loss of generality that $a < b$. If not, replace
$b$ with $b^r$, where $b^r > a$. Since $c_{b^r}(n) \ge c_{b}(n) / r$,
this replacement does not affect the order of growth.

Recall that a nonzero algebraic number $\alpha \in \mathbb{C}$ is a root of a unique irreducible
integer polynomial $P$ with positive leading coefficient and coprime coefficients.
The \emph{degree} of $\alpha$ is equal to the degree of $P$, and the \emph{height}
of $\alpha$ is the maximum of the absolute values of the cofficients of $P$.
A rational integer has degree $1$ and height equal to its absolute value.

Our proof relies on a version of Baker's theorem on linear forms in logarithms,
which we state without proof.

\begin{theorem} \cite[p. 23]{baker1975}
	Let $\alpha_1, \ldots, \alpha_n$ be nonzero algebraic numbers with degrees
	at most $d$. Suppose that $\alpha_1, \ldots, \alpha_{n-1}$ and $\alpha_n$
	have heights at most $A' \ge 4$ and $A \ge 4$ respectively.
	Let $\beta_1, \ldots, \beta_n$ be integers of absolute value at most $B \ge 2$.
	If
	\[
		\Lambda = \beta_1 \log \alpha_1 + \cdots + \beta_n \alpha_n \ne 0,
	\]
	there exists $C > 1$, depending only on $n$, $d$, and $A'$,
	such that
	\[
		|\Lambda| > C^{-\log A \log B}.
	\]
	See \cite{matveev2000} for an explicit lower bound.
	\label{baker}
\end{theorem}


Baker's theorem gives effective lower bounds on linear forms in logarithms
of algebraic numbers. In our context, it guarantees that the expression in
\eqref{linear-form} below cannot be too small, which limits how well powers of $a$
can be approximated by powers of $b$.

The next lemma deals with the possibility that some power of $a$ might be divisible by $b$.

\begin{lemma}
	Let $a, b \ge 2$ be integers with $\log(a)/\log(b)$ irrational.
	Then there exist $C$ and $C'$, depending only on $a$ and $b$,
	such that whenever
	\[
		a^n = b^r u,
	\]
	we have
	\[
		C n \le \log u \le C' n.
	\]
	In other words, $\log u \asymp n$.
	\label{lemma}
\end{lemma}

\begin{proof}
	By Lemma~\ref{lemma-1}, there exists a prime factor $p$ of $a$,
	and $C_1 > 0$ depending only on $a$ and $b$, such that 
	$\nu_p(u) > C_1 \log n$ for all $n$.
	Therefore, $u \ge p^{C_1 n}$ and $\log u \ge Cn$ for $C = C_1 \log p$.
	On the other hand, $\log u \le C' n$ for $C' = \log a$, since $u \le a^n$.
\end{proof}

We now prove the main theorem.

\begin{theorem}
	Let $a, b \ge 2$ be integers with $a < b$, and suppose that 
	$\log(a)/\log(b)$ is irrational.
	Then there exists $C > 0$, depending only on $a$ and $b$,
	such that
	\[
		c_{b}(a^n) > \frac{\log n}{\log \log n + C}
	\]
	holds for all sufficiently large $n$.
\end{theorem}

\begin{proof}
	Let
	\[
		a^n = b^m \left(d_1 b^{-m(1)} + \cdots + d_k b^{-m(k)}\right),
	\]
	where $m = \lceil \log_b a^n \rceil$, $k = c_{b}(a^n)$,
	$d_i \in \{1, \ldots, b-1\}$,
	and
	\[
		1 = m(1) < \cdots < m(k) \le m.
	\]
	That is, $d_1, \ldots, d_k$ are the nonzero digits of $a^n$ in base $b$,
	and $m(1), \ldots, m(k)$ are their positions when the digits are numbered
	from left (most significant) to right (least significant).
	We may assume that $k \ge 2$.

	Fix $i\in \{1,\dots,k-1\}$. We estimate the ratio $m(i+1)/m(i)$,
	which will ultimately bound $k$.

	Define
	\begin{align*}
		q & = b^{m(i)} (d_1 b^{-m(1)} + \cdots + d_i b^{-m(i)}),  \\
		r & = b^m (d_{i+1} b^{-m(i+1)} + \cdots + d_k b^{-m(k)}),
	\end{align*}
	so that
	\[
		a^n = b^{m - m(i)} q + r.
	\]

	From the base-$b$ expansion we obtain the bounds
	\begin{align*}
		b^{m-1}      & \le a^n \le b^m,            \\
		b^{m(i)-1}   & \le q   \le b^{m(i)},       \\
		b^{m-m(i+1)} & \le r   \le b^{m-m(i+1)+1}.
	\end{align*}

	Consequently,
	\begin{equation}
		b^{-m(i+1)} < a^{-n} r < b^{-m(i+1)+2},
	\end{equation}

	which implies that
	\begin{equation}
		\frac{m(i+1) - 2}{m(i)} \le
		\frac{-\log(a^{-n}r)}{\log q} \le
		\frac{m(i+1)}{m(i) - 1}
	\end{equation}
	provided that $q \ge 2$ and $m(i) \ge 2$.

	If $m(i) \ge 3$ (and $m(i+1) \ge 4$) then
	\begin{equation}
		\frac{m(i+1)-2}{m(i)} \ge \frac12\, \frac{m(i+1)}{m(i)},
		\qquad
		\frac{m(m+1)}{m(i)-1} \le \frac32\, \frac{m(i+1)}{m(i)},
	\end{equation}
	hence
	\begin{equation}
		\frac12\, \frac{m(i+1)}{m(i)} \le
		\frac{-\log(a^{-n}r)}{\log q} \le
		\frac32\, \frac{m(i+1)}{m(i)}.
		\label{estimate-1}
	\end{equation}

	Now set
	\begin{equation}
		\Lambda = \log(a^{-n} b^{m - m(i)} q) = -n \log a + (m - m(i)) \log b + \log q.
		\label{linear-form}
	\end{equation}
	Then (since $a^{-n} r < 1/2$)
	\[
		|\Lambda| = -\log(1 - a^{-n} r) < 2 a^{-n} r.
	\]

	We apply Baker's theorem to the linear form in \eqref{linear-form}.
	There are two cases, depending on the size of $q$.

	{\bf Case 1:} $q < b^2$, or equivalently, $m(i) \le 2$.

	In this case, we can apply Baker's theorem with 
	$A = b^2$ and $B = n$ to obtain
	\[
		|\Lambda| > C_1^{-\log n}
	\]
	for some constant $C_1 > 1$ depending only on $a$ and $b$.
	Thus,
	\[
		a^{-n} r > n^{-C_1},
	\]
	and hence, for $n$ sufficiently large,
	\[
		b^{-m(i+1)+2} > n^{-C_1},
	\]
	which implies that
	\[
		m(i+1) < C_2 \log n
	\]
	for some constant $C_{2}>0$ depending only on $a$ and $b$.
	
	Since $m(i) \ge 1$, we also have
	\[
		\frac{m(i+1)}{m(i)} < C_2 \log n.
	\]

	{\bf Case 2:} $q > b^2$, or equivalently, $m(i) \ge 3$.

	In this case, we can apply Baker's theorem with 
	$A = q$ and $B = n$ to obtain
	\[
		|\Lambda| > C_3^{-\log q \log n}
	\]
	for some $C_3>1$ depending only on $a$ and $b$. Thus,
	\[
		a^{-n} r > C_3^{-\log q \log n},
	\]
	and hence, for $n$ sufficiently large,
	\begin{equation}
		\frac{-\log(a^{-n} r)}{\log(q)} < C_4 \log n
		\label{estimate-2}
	\end{equation}
	for some $C_4 > 0$ depending only on $a$ and $b$.

	Combining \eqref{estimate-1} and \eqref{estimate-2} gives
	\[
		\frac{m(i+1)}{m(i)} < C_5 \log n.
		\label{m-ratio}
	\]
	for some $C_5 > 0$ depending only on $a$ and $b$.

	In either case, we have
	\[
		\frac{m(i+1)}{m(i)} < C_6 \log n, \qquad C_6 = \min(C_2, C_5).
	\]

	Summing the logarithms of these ratios,
	\[
		\log m(k) = \sum_{i=1}^{k-1} \log \left(\frac{m(i+1)}{m(i)}\right)
	\]
	which yields
	\begin{equation}
		\log m(k) < (k - 1) (\log \log n + C_7).
		\label{estimate-3}
	\end{equation}

	Write $a^n$ as $b^{m-m(k)} u$, where
	\[
		b^{m(k)-1} < u < b^{m(k)}.
	\]

	Thus $m(k) \asymp \log(u)$, and $\log(u) \asymp n$ by Lemma~\ref{lemma}, so
	\begin{equation}
		\log m(k) = \log n + O(1).
		\label{log-mk}
	\end{equation}
	Therefore, \eqref{estimate-3} and \eqref{log-mk} imply that
	\[
		k > \frac{\log n}{\log \log n + C}
	\]
	for all sufficiently large $n$, as required.
\end{proof}

\begin{thebibliography}{99}

	\bibitem{balog2012}
	Balog, A. and Dartyge, C.,
	``On the sum of the digits of multiples.''
	\textit{Moscow Journal of Combinatorics and Number Theory},
	\textbf{2} (2012), pp. 3--15.

	\bibitem{baker1975}
	Baker, A.,
	\textit{Transcendental Number Theory},
	Cambridge University Press, Cambridge, 1975.

	\bibitem{knuth1997}
	Knuth, D.\ E.,
	\textit{The Art of Computer Programming: Fundamental Algorithms}, Volume 1,
	Addison-Wesley Professional, 1997.

	\bibitem{matveev2000}
	Matveev, E.\ M.,
	``An explicit lower bound for a homogeneous rational linear form in the logarithms of algebraic numbers. II.''
	\textit{Izvestiya: Mathematics} \textbf{64.6} (2000), pp. 1217--1269.

	\bibitem{radcliffe2016}
	Radcliffe, D.,
	``The growth of digital sums of powers of 2,''
	arXiv preprint arXiv:1605.02839 (2016).

	\bibitem{sanna2015}
	Sanna, C.,
	``On the sum of digits of the factorial,''
	\textit{J. Number Theory} \textbf{147} (2015), pp. 836--841.

	\bibitem{senge-straus1973}
	Senge, H.\ G. and Straus, E.\ G.,
	``PV-numbers and sets of multiplicity,''
	\textit{Period. Math. Hungar.} \textbf{3} (1973), no.~1, pp. 93--100.

	\bibitem{sierpinski1970}
	Sierpi{\'n}ski, W.,
	\textit{250 Problems in Elementary Number Theory},
	American Elsevier Publishing Company, New York, 1970.

	\bibitem{stewart1980}
	Stewart, C.\ L.,
	``On the representation of an integer in two different bases,''
	\textit{J. Reine Angew. Math.} \textbf{319} (1980), pp. 63--72.

	\bibitem{stolarsky1980}
	Stolarsky, K.\ B.,
	``Integers whose multiples have anomalous digital frequencies,''
	\textit{Acta Arith} \textbf{38.2} (1980), pp. 117-128.
\end{thebibliography}

\end{document}
