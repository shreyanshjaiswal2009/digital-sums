\documentclass[12pt]{amsart} 
\usepackage{geometry}
\geometry{letterpaper}
\usepackage{amsmath}
\usepackage{amssymb}
\usepackage{tcolorbox}
\usepackage{hyperref}
\usepackage{biblatex}
\usepackage{xurl}
\usepackage{parskip}
\newtheorem{theorem}{Theorem}
\newtheorem{corollary}{Corollary}
\DeclareMathOperator{\lcm}{lcm}
\addbibresource{elementary-bounds-on-digital-sums-of-powers-radcliffe.bib}

\title{Elementary Bounds on Digital Sums of Powers, Factorials, and LCMs}
\author{David G Radcliffe}
\date{November 7, 2025}

\begin{document}

	\begin{abstract}
		We prove that the sum of the base-$b$ digits of $a^{n}$ grows at least logarithmically in $n$
		if $\log(d)/\log(b)$ is irrational, where $d$ is the smallest factor of $a$ 
		such that $\gcd(a/d, b) = 1$.
		Our approach uses only elementary number theory and applies to a wide class of sequences, 
		including factorials and $\Lambda(n) = \lcm(1, 2, \ldots, n)$.
	\end{abstract}

	\maketitle

	\section{Introduction}

		This article was inspired by the following problem, which was posed and solved
		by Wac{\l}aw Sierpi{\'n}ski\cite[Problem 209]{sierpinski1970}:
		\begin{quote}
			\emph{Prove that the sum of digits of the number $2^{n}$ (in decimal system)
			increases to infinity with $n$.}
		\end{quote}

		The reader is urged to try to solve this problem on their own before proceeding.
		Note that it is not enough to prove that the sum of digits of $2^{n}$ is
		unbounded, since the sequence is not monotonic.

		Consider the sequence of powers of $2$ (sequence \href{https://oeis.org/A000079}{A000079} 
		in the OEIS):

		\[
			1, 2, 4, 8, 16, 32, 64, 128, 256, 512, 1024, \ldots .
		\]

		This sequence grows very rapidly. Now, let us define another sequence, by adding
		the decimal digits of each power of $2$. For example, $16$ becomes $1+6=7$, and $32$
		becomes $3+2=5$. The first few terms of this new sequence 
		(\href{https://oeis.org/A001370}{A001370}) are listed below.

		\[
			1, 2, 4, 8, 7, 5, 10, 11, 13, 8, 7, \ldots .
		\]

		Our new sequence grows much more slowly, and it is not monotonic. 
		Nevertheless, it is reasonable to conjecture that it increases to infinity. 
		Indeed, one might guess that the sum of the decimal digits of $2^{n}$ 
		is asymptotic to $4.5 n \log_{10}2$, 
		since $2^{n}$ has $\lfloor n \log_{10}2\rfloor + 1$ decimal digits, 
		and the digits seem to be approximately uniformly distributed among 
		$0, 1, 2, \ldots, 9$. 
		However, this stronger conjecture remains to be proved.

		For an integer $b \ge 2$, we write $s_{b}(n)$ for the sum of the base-$b$
		digits of $n$, and $c_{b}(n)$ for the number of nonzero digits in that expansion.
		These functions are asymptotic to each other, since $c_{b}(n) \le s_{b}(n) \le
		(b-1) c_{b}(n)$ for all $n$ and $b$; so we will restrict our attention to
		$c_{b}(n)$. For a prime $p$, $\nu_{p}(n)$ denotes the exponent of $p$ in the
		prime factorization of $n$.

	\section{Digital sums of powers of two}

		We will present an informal proof that $\lim_{n\to\infty} c_{10}(2^{n}) = \infty$.
		The key idea, which might seem too obvious to state, is that a positive multiple 
		of a number cannot be smaller than that number.

		Let $n$ be a positive integer, and write the decimal expansion of $2^n$ as
		$2^n = \sum_{i=0}^{\infty} d_i 10^i$, where $d_i \in \{0, \ldots, 9\}$
		and $d_i = 0$ for all but finitely many terms. 
		Note that the final digit $d_0$ of $2^{n}$ cannot be zero,
		since $2^n$ is not divisible by $10$.

		If $2^{n} > 10$, then $2^{n}$ is divisible by 16; so
		$2^{n} \bmod 10^{4}$, the number formed by the last four digits of $2^{n}$, is
		also divisible by $16$. If the first three of these digits were zero, then $2^{n}
		\bmod 10^{4}$ would be less than $10$, which is impossible. 
		So at least one of the digits $d_1, d_2, d_3$ is nonzero.

		If $2^{n} > 10^4$, then $2^{n}$ is divisible by $2^{14}$; so
		$2^{n} \bmod 10^{14}$, the number formed by the last $14$ digits of $2^{n}$,
		is also divisible by $2^{14}$. If the first $10$ of these digits were zero,
		then $2^{n} \bmod 10^{14}$ would be less than $10^{4}$, which is impossible.
		So at least one of the digits $d_4, d_5, \ldots, d_{13}$ is nonzero.

		We can continue in this way, finding longer and longer non-overlapping blocks
		of digits, each containing at least one nonzero digit, as illustrated in Figure \ref{fig-blocks}.
		This shows that as $n$ increases to infinity, the number of nonzero digits 
		of $2^{n}$ also increases to infinity.
  
		\begin{figure}[h]
			$2^{103}$ = \framebox{$101412048018258352$}\framebox{$1197362564$}\framebox{$300$}\framebox{$8$}
			\caption{Each block contains at least one nonzero digit.}
			\label{fig-blocks}
		\end{figure}

		Let us formalize this argument.

		\begin{theorem}
			Let $(\varepsilon(k))_{k\ge1}$ be a sequence of integers such that 
			$\varepsilon(1) \ge 1$ and $2^{\varepsilon(k)} > 10^{\varepsilon(k-1)}$ for all $k \ge 2$. 
			If $n$ is a positive integer that is divisible by $2^{\varepsilon(k)}$ 
			but not divisible by $10$, then $c_{10}(n) \ge k$.
			\label{base-ten-lower-bound}
		\end{theorem}

		\begin{proof}
			The proof is by induction on $k$. The case $k = 1$ is trivial, so let us assume
			that $k \ge 2$. By the division algorithm, there exist integers $q \ge 0$ and
			$0 \le r < 10^{\varepsilon(k-1)}$ such that $n = 10^{\varepsilon(k-1)}q + r$.

			Since $n \ge 2^{\varepsilon(k)} > 10^{\varepsilon(k-1)}$, it follows that $q \ge 1$.

			Since $n$ and $10^{\varepsilon(k-1)}$ are divisible by $2^{\varepsilon(k-1)}$, 
			$r$ is also divisible by $2^{\varepsilon(k-1)}$. 
			But $r$ is not divisible by $10$,
			so $c_{10}(r) \ge k - 1$, by the induction hypothesis.

			Note that $c_{10}(n) = c_{10}(q) + c_{10}(r)$, since the digit expansion of $n$
			is the concatenation of the digit expansions of $q$ and $r$, possibly with
			leading zeros.

			Therefore, $c_{10}(n) \ge 1 + (k - 1) = k$.
		\end{proof}

		\begin{corollary}
				Let $a$ be a positive integer that is divisible by $2$ but not divisible by $10$.
				Then $c_{10}(a^n) \ge \log_4(n)$ for all $n \ge 1$.
			In particular,
			$\lim_{n\to \infty}c_{10}(a^{n}) = \infty$.
				\label{even-powers-base-ten-limit}
			\end{corollary}

		\begin{proof}
			Let $\varepsilon(k) = 4^{k-1}$ for $k \ge 1$. 
			This sequence satisfies $\varepsilon(1) \ge 1$
			and $2^{\varepsilon(k)} > 10^{\varepsilon(k-1)}$ for all $k \ge 2$.
			
			Let $n\ge 4$ be a positive integer, and let $k = \lceil \log_4 n \rceil$,
			so that $4^{k-1} < n \le 4^k$. Then $a^n$ is divisible by $2^n$,
			so $a^n$ is also divisible by $2^{\varepsilon(k)}$.
			
			Therefore,
			$c_{10}(a^n) \ge k \ge \log_4(n)$, by Theorem \ref{base-ten-lower-bound}.
		\end{proof}

	\section{Generalizing to other bases}

		Our proofs rely only on divisibility properties and therefore extend naturally
		to other bases.

		\begin{theorem}
			Let $b \ge 2$ be an integer that is not a power of a prime, and let $p$ be a
			prime divisor of $b$. Let $(\varepsilon(k))_{k\ge1}$ be a sequence of integers such
			that $\varepsilon(1) \ge 1$ and $p^{\varepsilon(k)} > b^{\varepsilon(k-1)}$ for all $k \ge 2$. 
			If $\nu_p(n) \ge \varepsilon(k)$ and $b \nmid n$ then $c_b(n) \ge k$.
			% If $n$ is a positive integer that is divisible by $p^{\varepsilon(k)}$ but not divisible by $b$, 
			% then $c_{b}(n) \ge k$.
			\label{general-powers-lower-bound}
		\end{theorem}

		\begin{proof}
			The proof is by induction on $k$. The case $k = 1$ is trivial, so let us assume
			that $k \ge 2$. By the division algorithm, there exist integers $q \ge 0$ and
			$0 \le r < b^{\varepsilon(k-1)}$ such that $n = b^{\varepsilon(k-1)}q + r$.

			Since $n \ge p^{\varepsilon(k)}> b^{\varepsilon(k-1)}$, it follows that $q \ge 1$.

			Since $n$ and $b^{\varepsilon(k-1)}$ are divisible by $p^{\varepsilon(k-1)}$, $r$ is also
			divisible by $p^{\varepsilon(k-1)}$. But $r$ is not divisible by $b$, so 
			$c_{b}(r) \ge k - 1$, by the induction hypothesis.

			Note that $c_{b}(n) = c_{b}(q) + c_{b}(r)$, since the digit expansion of $n$
			is the concatenation of the digit expansions of $q$ and $r$, possibly with
			leading zeros.

			Therefore, $c_{b}(n) \ge 1 + (k - 1) = k$.
		\end{proof}

		{\bf Notation.} Given an integer $b \ge 2$ with distinct prime divisors $p$ and $q$,
		define a function $\xi = \xi_{b,p,q}$ as follows:
			\[
				\xi(n) = \nu_{p}(n) - \nu_{q}(n) \frac{\nu_{p}(b)}{\nu_{q}(b)}.
			\]
		It is easy to verify that $\xi(b^r m) = \xi(t)$ for all $r \ge 0$
		and $t \ge 1$. Intuitively, $\xi$ is a modified $\nu_p$
		that ignores trailing zeros in the base-$b$ representation of its argument.

		\begin{theorem}
			Let $b \ge 2$ be an integer that is not a power of a prime, let $p$ and $q$
			be distinct prime divisors of $b$, and let 
			$(\varepsilon(k))_{k\ge1}$ be defined as in Theorem \ref{general-powers-lower-bound}.
			If $\xi(n) \ge \varepsilon(k)$, then $c_{b}(n) \ge k$. 
			In particular, $\lim_{n\to\infty} c_p(a_n) = \infty$
			for any sequence of positive integers that satisfies
			$\lim_{n\to\infty} \xi(a_n) = \infty$.
			\label{phi-lower-bound}
		\end{theorem}

		\begin{proof}
			Write $n$ as $b^{r} m$, where $m$ is not divisible by $b$. 
			Note that $\xi(n) = \xi(m)$. 
			Since $\nu_{p}(m) \ge \xi(m) \ge \varepsilon(k)$
			and $b \nmid m$, Theorem \ref{general-powers-lower-bound} implies that
			$c_{b}(m) \ge k$. But $c_{b}(m) = c_{b}(n)$, since $m$ and $n$ have the same
			digits in base $b$, apart from trailing zeros. Therefore, $c_{b}(n) \ge k$.
		\end{proof}

		\begin{theorem}
			Let $a \ge 2$ and $b \ge 2$ be integers. Let $d$ be the smallest factor of $a$
			such that $\gcd(a/d, b) = 1$.
			If $\log(d) / \log(b)$ is irrational, then 
			$c_{b}(a^{n}) \ge \log_r n$ for all $n \ge 1$, where $r \ge 2$ is an integer
			depending only on $a$ and $b$.
			In particular, $\lim_{n\to\infty}c_{b}(a^{n}) = \infty$.
			\label{general-digit-sum}
		\end{theorem}

		\begin{proof}
			Let $p_1^{e_1} \cdots p_t^{e_t}$ be the prime factorization of $b$.
			Then $d = p_1^{f_1} \cdots p_t^{f_t}$, where $f_i = \nu_{p_i}(a)$.
			Note that some of the $f_i$ may be zero.
			If 
			\[
				\frac{f_1}{e_1} = \cdots = \frac{f_t}{e_t}
			\]
			then $\log(d) / \log(b) = f_1/e_1$, which is rational.
			Therefore, if $\log(d) / \log(b)$ is irrational, then the ratios
			$f_i/e_i$ are not all equal, which implies that $b$ has two prime
			factors $p = p_i$ and $q = p_j$ such that $f_i / e_i > f_j / e_j$, 
			and so $\xi(a) > 0$.
		
			Let $r = \lceil \log(b) / \log(p) \rceil$, and let $\varepsilon(k) = r^{k-1}$ for $k \ge 1$.
			This sequence satisfies $\varepsilon(1) \ge 1$ and $p^{\varepsilon(k)} > b^{\varepsilon(k-1)}$
			for all $k \ge 2$.
			
			Let $n$ be a positive integer, and let $k = \lceil \log_r n \rceil$,
			so that $r^{k-1} < n \le r^k$. Then $a^n$ is divisible by $p^n$,
			so $a^n$ is also divisible by $p^{\varepsilon(k)}$. 
			
			Therefore, $c_{b}(a^n) \ge k \ge \log_r(n)$, by Theorem \ref{phi-lower-bound}.
		\end{proof}

		In 1973, Senge and Straus\cite[Theorem 3]{senge-straus1973}
			proved that if $a \ge 1$ and $b \ge 2$ are positive integers, then
			$\lim_{n \to \infty}c_{b}(a^{n}) = \infty$ if and only if
			$\log(a) / \log(b)$ is irrational. However, they did not demonstrate a lower bound. 
		In 1980, Stewart\cite[Theorem 2]{stewart1980} proved that
			if $\log(a)/\log(b)$ is irrational, then
			\[
				c_{b}(a^{n}) > \frac{\log n}{\log \log n + C} - 1
			\]
			for $n > 4$, where $C$ depends on $a$ and $b$ alone.
		Theorem \ref{general-digit-sum} achieves a stronger bound, 
		but with stricter conditions on $a$ and $b$.

	\section{Related sequences}

		Theorem \ref{phi-lower-bound} can be applied to many other sequences. 
		We will give two examples here. 
		As before, let $b \ge 2$ be an integer, 
		and let $p$ and $q$ be distinct prime divisors of $b$.

		\subsection*{Factorials}

			By Legendre's formula\cite[p. 263]{dickson1919},
			$\nu_{p}(n!) = (n - s_{p}(n)) / (p - 1)$. Thus,
			\[
				\xi(n!) \approx n \left( \frac{1}{p-1}- \frac{\nu_{p}(b)}{(q-1)\nu_{q}(b)}
				\right)
			\]
			whenever $(p-1) \nu_{p}(b) \ne (q-1) \nu_{q}(b)$, since $s_{p}(n) + s_{q}(n) \ll
			n$. 
			
			Therefore, $\lim_{n\to\infty}c_{b}(n!) = \infty$ for any $b$ with distinct
			factors $p$ and $q$ satisfying
			\[
				(p-1) \nu_{p}(b) > (q-1) \nu_{q}(b).
			\]
			In particular, $\lim_{n\to\infty}c_{10}(n!) = \infty$.

		\subsection*{Cumulative LCMs}

			Let $\Lambda_{n} = \lcm(1, 2, \ldots, n)$.
			This is sequence \href{https://oeis.org/A003418}{A003418} in the OEIS. 
			It is easy to see that $\nu_{p}(\Lambda_{n}) = \lfloor \log_{p}(n) \rfloor$. 
			Thus,
			\[
				\xi(\Lambda_{n}) = \lfloor \log_{p} n\rfloor - \lfloor \log_{q} n\rfloor
				\cdot \frac{\nu_{p}(b)}{\nu_{q}(b)}\approx \log n \cdot \left( \frac{1}{\log
				p}- \frac{1}{\log q}\cdot \frac{\nu_{p}(b)}{\nu_{q}(b)}\right).
			\]

			This quantity is nonzero, since $\log(p)/\log(q)$ is irrational;
			and we may assume that it is positive, else we can switch $p$ and $q$. Therefore,
			$\lim_{n\to\infty}\xi(\Lambda_{n}) = \infty$, hence $\lim_{n
			\to \infty}c_{b}(\Lambda_{n}) = \infty$ for every base $b$ that is not a prime
			power.

		These arguments imply lower bounds of the form $c_{b}(n!) > C \log n$
		and $c_b(\Lambda_n) > C \log \log n$, with certain restrictions on $b$.
		However, Sanna\cite{sanna2015} proved that 
		\[
			s_{b}(n!) > C \log n \log \log \log n
		\]
		for every integer $n > e^{e}$ and every $b \ge 2$, where $C$ is a constant 
		depending only on $b$; 
		and the same inequality for $s_{b}(\Lambda_n)$.

	\medskip

	\printbibliography

\end{document}