\documentclass[12pt]{amsart} 
\usepackage{geometry}
\geometry{letterpaper}
\usepackage{amsmath}
\usepackage{amssymb}
\usepackage{tcolorbox}
\usepackage{hyperref}
\usepackage{xurl}
\usepackage{parskip}
\newtheorem{theorem}{Theorem}
\newtheorem{corollary}{Corollary}
\newtheorem{lemma}{Lemma}
\theoremstyle{definition}
\newtheorem{exercise}{Exercise}
\DeclareMathOperator{\lcm}{lcm}

\title{Elementary Bounds on Digital Sums of Powers, Factorials, and LCMs}
\author{David G Radcliffe}
\date{November 17, 2025}

\begin{document}

	\begin{abstract}
		We prove that the sum of the base-$b$ digits of $a^{n}$ grows at least logarithmically in $n$
		if $\log(d)/\log(b)$ is irrational, where $d$ is the smallest factor of $a$ 
		such that $\gcd(a/d, b) = 1$.
		Our approach uses only elementary number theory and applies to a wide class of sequences, 
		including factorials and $\Lambda(n) = \lcm(1, 2, \ldots, n)$.
		We conclude with an expository proof of the previously known result that
		the sum of the base-$b$ digits of $a^{n}$ tends to infinity with $n$ if and only if
		$\log(a)/\log(b)$ is irrational.
	\end{abstract}

	\maketitle

	\section{Introduction}

		This article was inspired by the following problem, which was posed and solved
		by Wac{\l}aw Sierpi{\'n}ski\cite[Problem 209]{sierpinski1970}:
		\begin{quote}
			\emph{Prove that the sum of digits of the number $2^{n}$ (in decimal system)
			increases to infinity with $n$.}
		\end{quote}

		The reader is urged to attempt this problem on their own before proceeding.
		Note that it is not enough to prove that the sum of digits of $2^{n}$ is
		unbounded, since the sequence is not monotonic.

		Consider the sequence of powers of $2$ (sequence \href{https://oeis.org/A000079}{A000079} 
		in the OEIS):

		\[
			1, 2, 4, 8, 16, 32, 64, 128, 256, 512, 1024, \ldots .
		\]

		This sequence grows very rapidly. Now, let us define another sequence, by adding
		the decimal digits of each power of $2$. For example, $16$ becomes $1+6=7$, and $32$
		becomes $3+2=5$. The first few terms of this new sequence 
		(\href{https://oeis.org/A001370}{A001370}) are listed below.

		\[
			1, 2, 4, 8, 7, 5, 10, 11, 13, 8, 7, \ldots .
		\]

		Our new sequence grows much more slowly, and it is not monotonic. 
		Nevertheless, it is reasonable to conjecture that it tends to infinity. 
		Indeed, one might guess that the sum of the decimal digits of $2^{n}$ 
		is approximately equal to $4.5 n \log_{10}2$, 
		since $2^{n}$ has $\lfloor n \log_{10}2\rfloor + 1$ decimal digits, 
		and the digits seem to be approximately uniformly distributed among 
		$0, 1, 2, \ldots, 9$. 
		However, this stronger conjecture remains to be proved.

		For an integer $b \ge 2$, we write $s_{b}(n)$ for the sum of the base-$b$
		digits of $n$, and $c_{b}(n)$ for the number of nonzero digits in that expansion.
		These functions are asymptotic to each other, since $c_{b}(n) \le s_{b}(n) \le
		(b-1) c_{b}(n)$ for all $n$ and $b$; so we restrict our attention to
		$c_{b}(n)$. For a prime $p$, $\nu_{p}(n)$ denotes the exponent of $p$ in the
		prime factorization of $n$. If $p$ does not divide $n$ then $\nu_{p}(n) = 0$.

	\section{Digital sums of powers of two}

		We present an informal proof that $c_{10}(2^{n})$ tends to infinity as $n \to \infty$.
		The key idea, which might seem too obvious to state, is that a positive multiple 
		of a number cannot be smaller than that number.

		Let $n$ be a positive integer, and write the decimal expansion of $2^n$ as
		$2^n = \sum_{i=0}^{\infty} d_i 10^i$, where $d_i \in \{0, \ldots, 9\}$
		and $d_i = 0$ for all but finitely many terms. 
		Note that the final digit $d_0$ of $2^{n}$ cannot be zero,
		since $2^n$ is not divisible by $10$.

		If $2^{n} > 10$, then $2^{n}$ is divisible by 16; so
		$2^{n} \bmod 10^{4}$, the number formed by the last four digits of $2^{n}$, is
		also divisible by $16$. If the first three of these digits were zero, then $2^{n}
		\bmod 10^{4}$ would be less than $10$, which is impossible. 
		So at least one of the digits $d_1, d_2, d_3$ is nonzero.

		If $2^{n} > 10^4$, then $2^{n}$ is divisible by $2^{14}$; so
		$2^{n} \bmod 10^{14}$, the number formed by the last $14$ digits of $2^{n}$,
		is also divisible by $2^{14}$. If the first $10$ of these digits were zero,
		then $2^{n} \bmod 10^{14}$ would be less than $10^{4}$, which is impossible.
		So at least one of the digits $d_4, d_5, \ldots, d_{13}$ is nonzero.

		We can continue in this way, finding longer and longer non-overlapping blocks
		of digits, each containing at least one nonzero digit, as illustrated in Figure \ref{fig-blocks}.
		This proves that $c_{10}(2^n)$ tends to infinity as $n \to \infty$.
  
		\begin{figure}[h]
			$2^{103}$ = \framebox{$101412048018258352$}\framebox{$1197362564$}\framebox{$300$}\framebox{$8$}
			\caption{Each block contains at least one nonzero digit.}
			\label{fig-blocks}
		\end{figure}

		Let us formalize this argument.

		\begin{theorem}
			Let $(\varepsilon(k))_{k\ge1}$ be a sequence of integers such that 
			$\varepsilon(1) \ge 1$ and $2^{\varepsilon(k)} > 10^{\varepsilon(k-1)}$ for all $k \ge 2$. 
			If $n$ is a positive integer that is divisible by $2^{\varepsilon(k)}$ 
			but not divisible by $10$, then $c_{10}(n) \ge k$.
			\label{base-ten-lower-bound}
		\end{theorem}

		\begin{proof}
			The proof is by induction on $k$. The case $k = 1$ is trivial, so let us assume
			that $k \ge 2$. 
			By the division algorithm, there exist integers $q$ and $r$ such that
			\[
				n = 10^{\varepsilon(k-1)} q + r,
			\]
			where $q \ge 0$ and $0 \le r < 10^{\varepsilon(k-1)}$.
			

			Since $n \ge 2^{\varepsilon(k)} > 10^{\varepsilon(k-1)}$, it follows that $q \ge 1$.

			Since $n$ and $10^{\varepsilon(k-1)}$ are divisible by $2^{\varepsilon(k-1)}$, 
			the remainder $r$ is also divisible by $2^{\varepsilon(k-1)}$. 
			But $r$ is not divisible by $10$,
			so $c_{10}(r) \ge k - 1$, by the induction hypothesis.

			Note that $c_{10}(n) = c_{10}(q) + c_{10}(r)$, since the digit expansion of $n$
			is the concatenation of the digit expansions of $q$ and $r$, possibly with
			leading zeros.

			Therefore, $c_{10}(n) \ge 1 + (k - 1) = k$.
		\end{proof}

		\begin{corollary}
			Let $a$ be a positive integer that is divisible by $2$ but not divisible by $10$.
			Then $c_{10}(a^n) \ge \log_4(n)$ for all $n \ge 1$.
			In particular,
			$\lim_{n\to \infty}c_{10}(a^{n}) = \infty$.
				\label{even-powers-base-ten-limit}
			\end{corollary}

		\begin{proof}
			Let $\varepsilon(k) = 4^{k-1}$ for $k \ge 1$. 
			This sequence satisfies $\varepsilon(1) \ge 1$
			and 
			\[
				2^{\varepsilon(k)} > 10^{\varepsilon(k-1)}
			\]
			for all $k \ge 2$.
			
			Let $n\ge 4$ be a positive integer, and let $k = \lceil \log_4 n \rceil$,
			so that $4^{k-1} < n \le 4^k$. Then $a^n$ is divisible by $2^n$,
			so $a^n$ is also divisible by $2^{\varepsilon(k)}$.
			But $a^n$ is not divisible by $10$.
			
			Therefore,
			$c_{10}(a^n) \ge k \ge \log_4(n)$, by Theorem \ref{base-ten-lower-bound}.
		\end{proof}

		\begin{exercise}
			Show that every power of 3 has a multiple $m$, not divisible by $10$,
			such that $c_{10}(m) = 2$.
		\end{exercise}

	\section{Generalizing to other bases}

		Our proofs rely only on divisibility properties and therefore extend naturally
		to other bases.

		\begin{theorem}
			Let $b \ge 2$ be an integer that is not a power of a prime, and let $p$ be a
			prime divisor of $b$. Let $(\varepsilon(k))_{k\ge1}$ be a sequence of integers such
			that $\varepsilon(1) \ge 1$ and $p^{\varepsilon(k)} > b^{\varepsilon(k-1)}$ for all $k \ge 2$. 
			If $\nu_p(n) \ge \varepsilon(k)$ and $b \nmid n$ then $c_b(n) \ge k$.
			\label{general-powers-lower-bound}
		\end{theorem}

		\begin{proof}
			The proof is by induction on $k$. The case $k = 1$ is trivial, so let us assume
			that $k \ge 2$. By the division algorithm, there exist integers 
			$q$ and $r$ such that
			\[
				n = b^{\varepsilon(k-1)}q + r,
			\]
			where $q \ge 0$ and $0 \le r < b^{\varepsilon(k-1)}$.

			Since $n \ge p^{\varepsilon(k)}> b^{\varepsilon(k-1)}$, it follows that $q \ge 1$.

			Since $n$ and $b^{\varepsilon(k-1)}$ are divisible by $p^{\varepsilon(k-1)}$, 
			the remainder $r$ is also divisible by $p^{\varepsilon(k-1)}$. 
			But $r$ is not divisible by $b$, 
			so $c_{b}(r) \ge k - 1$, by the induction hypothesis.

			Note that $c_{b}(n) = c_{b}(q) + c_{b}(r)$, since the digit expansion of $n$
			is the concatenation of the digit expansions of $q$ and $r$, possibly with
			leading zeros.

			Therefore, $c_{b}(n) \ge 1 + (k - 1) = k$.
		\end{proof}

		{\bf Notation.} Given an integer $b \ge 2$ with distinct prime divisors $p$ and $q$,
		define a function $\xi = \xi_{b,p,q}$ as follows:
			\[
				\xi(n) = \nu_{p}(n) - \nu_{q}(n) \frac{\nu_{p}(b)}{\nu_{q}(b)}.
			\]
		The reader can verify that $\xi(b^r u) = \xi(u)$ for all $r \ge 0$
		and $u \ge 1$. Intuitively, $\xi$ is a modified $\nu_p$
		that ignores trailing zeros in the base-$b$ representation of its argument.

		\begin{theorem}
			Let $b \ge 2$ be an integer that is not a power of a prime, let $p$ and $q$
			be distinct prime divisors of $b$, and let 
			$(\varepsilon(k))_{k\ge1}$ be defined as in Theorem \ref{general-powers-lower-bound}.
			If $\xi(n) \ge \varepsilon(k)$, then $c_{b}(n) \ge k$. 
			In particular, $\lim_{n\to\infty} c_b(a_n) = \infty$
			for any sequence of positive integers that satisfies
			$\lim_{n\to\infty} \xi(a_n) = \infty$.
			\label{phi-lower-bound}
		\end{theorem}

		\begin{proof}
			Write $n$ as $b^{r} u$, where $u$ is not divisible by $b$. 
			Note that $\xi(n) = \xi(u)$. 
			Since $\nu_{p}(u) \ge \xi(u) \ge \varepsilon(k)$
			and $b \nmid u$, Theorem \ref{general-powers-lower-bound} implies that
			$c_{b}(u) \ge k$. But $c_{b}(u) = c_{b}(n)$, since $u$ and $n$ have the same
			digits in base $b$, apart from trailing zeros. Therefore, $c_{b}(n) \ge k$.
		\end{proof}

		\begin{theorem}
			Let $a \ge 2$ and $b \ge 2$ be integers. Let $d$ be the smallest factor of $a$
			such that $\gcd(a/d, b) = 1$, and suppose that $\log(d) / \log(b)$ is irrational.
			Then $c_{b}(a^{n}) > C \log n$ for all $n \ge 1$, where $C > 0$ depends
			only on $a$ and $b$.
			In particular, $\lim_{n\to\infty}c_{b}(a^{n}) = \infty$.
			\label{general-digit-sum}
		\end{theorem}

		\begin{proof}
			Let $p_1^{e_1} \cdots p_t^{e_t}$ be the prime factorization of $b$.
			Then
			\[
				d = p_1^{f_1} \cdots p_t^{f_t},
			\]
			where $f_i = \nu_{p_i}(a)$.
			Note that some of the $f_i$ may be zero.
			If 
			\[
				\frac{f_1}{e_1} = \cdots = \frac{f_t}{e_t}
			\]
			then $\log(d) / \log(b) = f_1/e_1$, which is rational.
			Therefore, if $\log(d) / \log(b)$ is irrational, then the ratios
			$f_i/e_i$ are not all equal, which implies that $b$ has two prime
			factors $p = p_i$ and $q = p_j$ such that
			\[
				\frac{f_i}{e_i} > \frac{f_j}{e_j},
			\]
			and so $\xi(a) > 0$.

			Let $r = \lceil \log_p b \rceil$, 
			and let $\varepsilon(k) = r^{k-1}$ for $k \ge 1$.
			This sequence satisfies $\varepsilon(1) \ge 1$ and
			\[
				p^{\varepsilon(k)} > b^{\varepsilon(k-1)}
			\]
			for all $k \ge 2$.
			
			Let $n$ be a positive integer, and let $k = \lceil \log_r \xi(a^n) \rceil$,
			so that
				$r^{k-1} < \xi(a^n) \le r^k$.
			Then $\xi(a^n) > \varepsilon(k)$, so Theorem \ref{phi-lower-bound} implies that $c_p(a^n) \ge k$.
			But $\xi(a^n) = n \xi(a)$, hence $k = \Theta(\log n)$, and
			the conclusion follows.
		\end{proof}

		In 1973, Senge and Straus\cite[Theorem 3]{senge-straus1973}
		proved that if $a \ge 1$ and $b \ge 2$ are positive integers, then
		$\lim_{n \to \infty}c_{b}(a^{n}) = \infty$ if and only if
		$\log(a) / \log(b)$ is irrational. However, they did not demonstrate a lower bound. 
		In 1980, Stewart\cite[Theorem 2]{stewart1980} proved that
		if $\log(a)/\log(b)$ is irrational, then
		\[
			c_{b}(a^{n}) > \frac{\log n}{\log \log n + C} - 1
		\]
		for $n > 4$, where $C$ depends on $a$ and $b$ alone.
		Theorem \ref{general-digit-sum} achieves a stronger bound, 
		but with stricter conditions on $a$ and $b$.
		We replicate Stewart's result in Section \ref{general-case}.

	\section{Related sequences}

		Theorem \ref{phi-lower-bound} can be applied to many other sequences. 
		We give two examples here. 

		\begin{theorem}
			Let $b \ge 2$ be an integer. Suppose that $b$ has prime divisors $p$ and $q$
			such that
			\[
				(p - 1) \nu_{p}(b) \ne (q-1) \nu_{q}(b).
			\]
			Then $c_b(n!) > C \log n$ for some $C > 0$ depending only on $b$.
			In particular, $c_b(n!) \to \infty$ as $n \to \infty$.
		\end{theorem}
		
		\begin{proof}
			Assume without loss of generality that
			$(p - 1) \nu_{p}(b) < (q-1) \nu_{q}(b)$.
			This condition may be rewritten as
			\[
				\frac{1}{p-1} - \frac{\nu_p(b)}{(q - 1) \nu_q(b)} > 0.
			\]

			By Legendre's formula\cite[p. 263]{dickson1919},
			\[
				\nu_{p}(n!) = \frac{n - s_{p}(n)}{p - 1}.
			\]
			Since $s_{p}(n)/n \to 0$ and $s_{q}(n)/n \to 0$ as $n \to \infty$,
			\[
				\xi(n!) = n \left( \frac{1}{p-1}- \frac{\nu_{p}(b)}{(q-1)\nu_{q}(b)}
				+ o(1) \right) = \Theta(n).
			\]
			Therefore, by Theorem \ref{phi-lower-bound},
			\[
				c_b(n!) > C \log n
			\]
			for some $C > 0$ depending only on $b$.
		\end{proof}

		\begin{theorem}
			Let $\Lambda_n = \lcm(1, 2, \ldots, n)$, and let $b \ge 2$ be an integer
			that is not a power of a prime. Then $c_b(\Lambda_n) > C \log \log n$
			for some $C > 0$ depending only on $b$. In particular, $c_b(\Lambda_n) \to \infty$
			as $n \to \infty$.
		\end{theorem}

		\begin{proof}
			Let $p$ and $q$ be two distinct divisors of $b$.
			It is well known \cite[p. 62]{andrica2020} that $\nu_{p}(\Lambda_{n}) = \lfloor \log_{p}(n) \rfloor$.
			Thus,
			\[
				\xi(\Lambda_{n}) = \lfloor \log_{p} n\rfloor - \lfloor \log_{q} n\rfloor
				\cdot \frac{\nu_{p}(b)}{\nu_{q}(b)} = \log n \cdot \left( \frac{1}{\log
				p}- \frac{1}{\log q}\cdot \frac{\nu_{p}(b)}{\nu_{q}(b)}\right) + O(1).
			\]

			Since $\log(p)/\log(q)$ is irrational,
			\[
				\frac{1}{\log p}- \frac{1}{\log q}\cdot \frac{\nu_{p}(b)}{\nu_{q}(b)} \ne 0,
			\]
			and we may assume without loss of generality that it is positive. 

			Therefore,
			\[
			   \xi(\Lambda_n) = \Theta(\log n),
			\]
			so by Theorem \ref{phi-lower-bound},
			\[
				c_b(\Lambda_n) > C \log \log n
			\]
			for some $C > 0$ depending only on $b$.

		\end{proof}

		Stronger bounds are known.
		In 2015, Sanna\cite{sanna2015} proved that 
		\[
			s_{b}(n!) > C \log n \log \log \log n
		\]
		for every integer $n > e^{e}$ and every $b \ge 2$, where $C$ is a constant 
		depending only on $b$,
		and they established the same lower bound for $s_{b}(\Lambda_n)$.

	\section{Digital sums of powers: the general case}
		\label{general-case}

		In this final section, we will prove that the number of nonzero digits in the base-$b$
		expansion of $a^n$ tends to infinity as $n \to \infty$, provided that $\log(b)/\log(a)$
		is irrational. This result appears in earlier work of Senge–Straus \cite{senge-straus1973} 
		and Stewart \cite{stewart1980}, but we present an argument that we hope is more accessible.

		The irrationality condition is necessary. Indeed, if $\log(a)/\log(b) = r/s \in \mathbb{Q}$, then
		\[
			a^{ns} = b^{nr}
		\]
		for every integer $n$, so $a^{ns}$ has only one nonzero digit in base $b$.

		Recall that a nonzero algebraic number $\alpha \in \mathbb{C}$ is a root of a unique irreducible 
		integer polynomial $P$ with positive leading coefficient and coprime coefficients.
		The \emph{degree} of $\alpha$ is equal to the degree of $P$, and the \emph{height} 
		of $\alpha$ is the maximum of the absolute values of the cofficients of $P$. 
		A rational integer has degree $1$ and height equal to its absolute value.

		Our proof relies on a version of Baker's theorem on linear forms in logarithms, 
		which we use as a black box.

		\begin{theorem} \cite[p. 23]{baker1975}
			Let $\alpha_1, \ldots, \alpha_n$ be nonzero algebraic numbers with degrees
			at most $d$. Suppose that $\alpha_1, \ldots, \alpha_{n-1}$ and $\alpha_n$
			have heights at most $A' \ge 4$ and $A \ge 4$ respectively. 
			Let $\beta_1, \ldots, \beta_n$ be integers of absolute value at most $B \ge 2$.
			If
			\[
				\Lambda = \beta_1 \log \alpha_1 + \cdots + \beta_n \alpha_n \ne 0,
			\]
			there exists $C > 1$, depending only on $n$, $d$, and $A'$,
			such that 
			\[
				|\Lambda| > C^{-\log A \log B}.
			\]
			\label{baker}
		\end{theorem}

		Baker's theorem gives effective lower bounds on linear forms in logarithms
		of algebraic numbers. In our context, it guarantees that the expression in
		Equation \ref{linear-form} (below) cannot be too small, which limits how well powers of $a$
		can be approximated by powers of $b$.

		The next lemma deals with the possibility that some power of $a$ might be divisible by $b$.
		
		\begin{lemma}
			Let $a, b \ge 2$ be integers with $\log(a)/\log(b)$ irrational.
			Then there exist $C$ and $C'$, depending only on $a$ and $b$, 
			such that whenever
			\[
				a^n = b^r u,
			\]
			we have 
			\[
				C n \le \log u \le C' n.
			\]
			In other words, $\log u = \Theta(n)$.
			\label{lemma}
		\end{lemma}

		\begin{proof}
			Since $\log(a)/\log(b)$ is irrational, we can choose primes $p, q$ such that
			\[
				\nu_p(a) \nu_q(b) > \nu_q(a) \nu_p(b).
			\]
			Hence $\xi(a)>0$ for the function $\xi=\xi_{b,p,q}$ defined in Section 3.

			Because $\nu_p(u) \ge \xi(u) = \xi(a^n) = n \xi(a)$, the integer $u$ is 
			divisible by $p^{\lceil n\xi(a)\rceil}$. Therefore
			\[
				\log u \ge Cn, \qquad C = \xi(a) \log p > 0.
			\]

			On the other hand, $\log u \le C' n$ for $C' = \log a$, since $u \le a^n$.
		\end{proof}

		We now prove the main theorem.

		\begin{theorem} 
			Let $a, b \ge 2$ be integers, and suppose that $\log(a)/\log(b)$ is irrational.
			Then for all sufficiently large $n$,
			\[
				c_b(a^n) > \frac{\log n}{\log \log n + C}
			\]
			for some $C > 0$ depending only on $a$ and $b$.
			In particular,
			$c_b(a^n) \to \infty$.
		\end{theorem}

		\begin{proof}
			Let
			\begin{equation*}
				a^n = b^m \left(d_i b^{-m(1)} + \cdots + d_k b^{-m(k)}\right),
			\end{equation*}
			where $m = \lceil \log_b a \rceil$, $k = c_b(a^n)$, 
			$d_i \in \{1, \ldots, b-1\}$,
			and
			\[
				1 = m(1) < \cdots < m(k) \le m.
			\]
			That is, $d_1, \ldots, d_k$ are the nonzero digits of $a^n$ in base $b$,
			and $m(1), \ldots, m(k)$ are their positions, counting from
			the left.
			Assume that $k \ge 2$.

			Fix $i\in \{1,\dots,k-1\}$. We estimate the ratio $m(i+1)/m(i)$, 
			which will ultimately bound $k$.

			Define
			\begin{align*}
				q &= b^{m(i)} (d_1 b^{-m(1)} + \cdots + d_i b^{-m(i)}), \\
				r &= b^m (d_{i+1} b^{-m(i+1)} + \cdots + d_k b^{-m(k)}),
			\end{align*}
			so that
			\[
				a^n = b^{m - m(i)} q + r.
			\]

			From the base-$b$ expansion we obtain the bounds
			\begin{align*}
			    b^{m-1}      &\le a^n \le b^m, \\
                b^{m(i)-1}   &\le q   \le b^{m(i)}, \\
				b^{m-m(i+1)} &\le r   \le b^{m-m(i+1)+1}.
			\end{align*}

			These imply the approximation
			\begin{equation}
				\frac{-\log(a^{-n} r)}{\log(q)} = \frac{m(i+1)}{m(i)} + O(1).
				\label{estimate-1}
			\end{equation}

			Now set
			\begin{equation}
			   \Lambda = \log(a^{-n} b^{m - m(i)} q) = -n \log a + (m - m(i)) \log b + \log q.
			   \label{linear-form}
			\end{equation}
			Then (since $a^{-n} r < 1/2$)
			\[
				|\Lambda| = -\log(1 - a^{-n} r) < 2 a^{-n} r.
			\]
			By Theorem \ref{baker}, 
			\[
				|\Lambda| > C^{-\log q \log n}
			\]
			for some $C>1$ depending only on $a$ and $b$. Thus, 
			\[
				a^{-n} r > C^{-\log q \log n},
			\]
			and hence, for $n$ sufficiently large,
			\begin{equation}
				\frac{-\log(a^{-n} r)}{\log(q)} < C \log n
				\label{estimate-2}
			\end{equation}
			for some $C > 0$ depending only on $a$ and $b$.

			Combining Equations \ref{estimate-1} and \ref{estimate-2} gives
			\begin{equation}
				\frac{m(i+1)}{m(i)} < C \log n.
				\label{m-ratio}
			\end{equation}

			Summing the logarithms of these ratios,
			\[
				\log m(k) = \sum_{i=1}^{k-1} \log \left(\frac{m(i+1)}{m(i)}\right)		          
			\]
			which yields
			\begin{equation}
			   \log m(k) < (k - 1) (\log \log n + C).
			   \label{estimate-3}
			\end{equation}

			Write $a^n$ as $b^{m-m(k)} u$, where 
			\[
				b^{m(k)-1} < u < b^{m(k)}.
			\]

			Thus $m(k) = \Theta(\log(u))$, and $\log(u) = \Theta(n)$ by Lemma~\ref{lemma}, so
			\[
				\log m(k) = \log n + O(1).
			\]
			Therefore, Equation \eqref{estimate-3} implies that		
			\[
				k > \frac{\log n}{\log \log n + C}
			\]
			for all sufficiently large $n$, as required.		
		\end{proof}
	
	\begin{exercise}
		Prove that the number of nonzero decimal digits in the $n$th
		Fibonacci number tends to infinity as $n \to \infty$.
	\end{exercise}

\begin{thebibliography}{99}

\bibitem{andrica2020}
Andrica, D., R\u{a}dulescu, S., and {\c{T}}urca{\c{s}}, G.\ C.,
``The exponent of a group: Properties, computations and applications,''
in \textit{Discrete Mathematics and Applications},
A.\ M.\ Raigorodskii and M.\ Th.\ Rassias (eds.),
Springer, Cham, 2020, pp.~57--108.

\bibitem{baker1975}
Baker, A.,
\textit{Transcendental Number Theory},
Cambridge University Press, Cambridge, 1975.

\bibitem{dickson1919}
Dickson, L.\ E.,
\textit{History of the Theory of Numbers}, Vol.~1,
Carnegie Institution of Washington, 1919.

\bibitem{sanna2015}
Sanna, C.,
``On the sum of digits of the factorial,''
\textit{J. Number Theory} \textbf{147} (2015), 836--841.

\bibitem{senge-straus1973}
Senge, H.\ G. and Straus, E.\ G.,
``PV-numbers and sets of multiplicity,''
\textit{Period. Math. Hungar.} \textbf{3} (1973), no.~1, 93--100.

\bibitem{sierpinski1970}
Sierpi{\'n}ski, W.,
\textit{250 Problems in Elementary Number Theory},
American Elsevier Publishing Company, New York, 1970.

\bibitem{stewart1980}
Stewart, C.\ L.,
``On the representation of an integer in two different bases,''
\textit{J. Reine Angew. Math.} \textbf{319} (1980), 63--72.

\end{thebibliography}

\end{document}
