% !TEX TS-program = pdflatex
% !TEX encoding = UTF-8 Unicode

% This is a simple template for a LaTeX document using the "article" class.
% See "book", "report", "letter" for other types of document.

\documentclass[11pt]{amsart} % use larger type; default would be 10pt

\usepackage[utf8]{inputenc} % set input encoding (not needed with XeLaTeX)

%%% Examples of Article customizations
% These packages are optional, depending whether you want the features they provide.
% See the LaTeX Companion or other references for full information.

%%% PAGE DIMENSIONS
\usepackage{geometry} % to change the page dimensions
\geometry{letterpaper} % or letterpaper (US) or a5paper or....
% \geometry{margin=2in} % for example, change the margins to 2 inches all round
% \geometry{landscape} % set up the page for landscape
%   read geometry.pdf for detailed page layout information

\usepackage{graphicx} % support the \includegraphics command and options
% \usepackage[parfill]{parskip} % Activate to begin paragraphs with an empty line rather than an indent

%%% PACKAGES
\usepackage{csquotes}
\usepackage{booktabs} % for much better looking tables
\usepackage{array} % for better arrays (eg matrices) in maths
\usepackage{paralist} % very flexible & customisable lists (eg. enumerate/itemize, etc.)
\usepackage{verbatim} % adds environment for commenting out blocks of text & for better verbatim
\usepackage{subfig} % make it possible to include more than one captioned figure/table in a single float
\usepackage{amsmath}
\usepackage{amssymb}
\usepackage{tcolorbox}
\usepackage{sistyle}
\SIthousandsep{,}
\usepackage{hyperref}
% These packages are all incorporated in the memoir class to one degree or another...

%%% HEADERS & FOOTERS
%\usepackage{fancyhdr} % This should be set AFTER setting up the page geometry
%\pagestyle{fancy} % options: empty , plain , fancy
%\renewcommand{\headrulewidth}{0pt} % customise the layout...
%\lhead{}\chead{}\rhead{}
%\lfoot{}\cfoot{\thepage}\rfoot{}

%%% SECTION TITLE APPEARANCE
%\usepackage{sectsty}
%\allsectionsfont{\sffamily\mdseries\upshape} % (See the fntguide.pdf for font help)
% (This matches ConTeXt defaults)

%%% ToC (table of contents) APPEARANCE
%\usepackage[nottoc,notlof,notlot]{tocbibind} % Put the bibliography in the ToC
%\usepackage[titles,subfigure]{tocloft} % Alter the style of the Table of Contents
%\renewcommand{\cftsecfont}{\rmfamily\mdseries\upshape}
%\renewcommand{\cftsecpagefont}{\rmfamily\mdseries\upshape} % No bold!


%%% THEOREM ENVIRONMENTS
\newtheorem{theorem}{Theorem}
\newtheorem{corollary}{Corollary}[theorem]

\usepackage{biblatex} %Imports biblatex package
\usepackage{xurl}
\addbibresource{growth-of-digital-sums-of-powers.bib} %Import the bibliography file


%%% END Article customizations

%%% The "real" document content comes below...

\title{The growth of digital sums of powers}
\author{David G Radcliffe}
\date{} % Activate to display a given date or no date (if empty),
         % otherwise the current date is printed 

 
\begin{document}

\begin{abstract}
We establish sufficient conditions for the sum of the base-$b$ digits of $a^n$ to diverge to infinity, 
and prove that under mild hypotheses this sum grows at least logarithmically. Our approach uses only 
elementary number-theoretic arguments and applies to a wide class of sequences, including factorials 
and least common multiples.
\end{abstract}

\maketitle

\section{Motivation}

This article was inspired by the following problem, due to Wacław Sierpiński:
\begin{quote}
\emph{Prove that the sum of digits of the number $2^n$ (in decimal system) increases to infinity with $n$.}
\cite[Problem 209]{sierpinski1970}
\end{quote}

% \begin{quotation}
% Prove that the sum of digits of the number $2^n$ (in decimal system) increases
% to infinity with $n$. \cite[Problem 209]{sierpinski1970}
% \end{quotation}

We will prove a sufficient condition on positive integers $a$ and $b$ which implies that 
the sum of the base-$b$ digits of $a^n$ grows at least logarithmically in $n$.
The behavior of digital sums of powers has been studied by Senge and Straus (1973) and Stewart (1980), 
but their results rely on deeper tools from transcendence theory. In contrast, the arguments here are elementary.

Consider the sequence of powers of 2:

$$1, 2, 4, 8, 16, 32, 64, 128, 256, 512, 1024, \ldots.$$

This sequence grows very rapidly. Now, let us define another sequence by adding the decimal digits of each power of 2. For example,
16 becomes $1+6=7$, and 128 becomes $1+2+8=11$. The first few terms of this new sequence are listed below.

$$1, 2, 4, 8, 7, 5, 10, 11, 13, 8, 7, \ldots$$

It is apparent that this sequence grows much more slowly, and it is not monotone.
Nevertheless, it is reasonable to conjecture that the sequence diverges to infinity. Indeed, we should
expect that the sum of the decimal digits of $2^n$ is approximately $4.5 n \log_{10} 2$ when $n$ is large,
since $2^n$ has $\lfloor n \log_{10} 2\rfloor + 1$ decimal digits and the digits seem to be approximately uniformly
distributed among $0, 1, 2, \ldots, 9$. However, this stronger statement remains to be proved.

For an integer $b \ge 2$, we write $s_b(n)$ for the sum of the base-$b$ digits of $n$, 
and $c_b(n)$ for the number of nonzero digits in that expansion. 
These functions are asymptotic to each other,
since $c_b(n) \le s_b(n) \le (b-1) c_b(n)$ for all $n$ and $b$, 
so we will restrict our attention to $c_b(n)$.
For a prime $p$, $\nu_p(n)$ denotes the exponent of $p$ in the prime factorization of $n$.

Let us prove that $\lim\limits_{n\to\infty} c_{10}(2^n) = \infty$.
Observe that the final digit of $2^n$ cannot be 0, since only multiples of 10 can end in 0.

If $2^n$ has four or more digits, then the last four digits cannot start with three consecutive zeros.
This is because $2^n$ is divisible by 16, so $2^n \bmod 10^4$, 
the number formed by the last four digits of $2^n$,
is also divisible by 16. If the first three of these digits were zero, 
then $2^n \bmod 10^4$ would be less than 10,
which is a contradiction.

Similarly, if $2^n$ has 14 or more digits, then $2^n \bmod 10^{14} \ge 2^{14} > 10^4$.
So the last 14 digits of $2^n$ cannot start with 10 consecutive zeros.

We can continue in this way, finding longer and longer non-overlapping blocks of digits 
which cannot all be zeros.
This shows that as $n$ increases to infinity, the number of nonzero digits of $2^n$ 
also increases to infinity.
Let us formalize this argument.

\begin{theorem}
  Let $\{e_k\}$ be a sequence of positive integers such that $e_1 = 1$ 
  and $2^{e_{k+1}} > 10^{e_k}$ for all $k \ge 1$.
  If $n$ is a positive integer that is divisible by $2^{e_k}$ but not divisible by 10, 
  then $c_{10}(n) \ge k$.
  \label{base-ten-lower-bound}
\end{theorem}

\begin{proof}
The proof is by induction on $k$. The case $k = 1$ is trivial, so let us assume that $k \ge 2$.
By the division algorithm, there exist integers $q \ge 0$ and $0 \le r < 10^{e_{k-1}}$
such that
$$ n = 10^{e_{k-1}} q + r.$$

Since $n \ge 2^{e_k} > 10^{e_{k-1}}$, it follows that $q \ge 1$.

Since $n$ and $10^{e_{k-1}}$ are divisible by $2^{e_{k-1}}$, 
$r$ is also divisible by $2^{e_{k-1}}$.
Moreover, $r$ is not divisible by 10, so 
$c_{10}(r) \ge k - 1 $ by the induction hypothesis.

Note that $c_{10}(n) = c_{10}(q) + c_{10}(r)$, since the digit expansion of $n$ is the
concatenation of the digit expansions of $q$ and $r$, possibly with leading zeros. Therefore,
$$
c_{10}(n)  = c_{10}(q) + c_{10}(r) 
        \ge 1 + (k - 1) 
        = k.
$$
\end{proof}
  
\begin{corollary}
Let $a$ be a positive integer that is divisible by 2 but not divisible by 10.
Then $\lim\limits_{n\to \infty} c_{10}(a^n) = \infty$.
\label{even-powers-base-ten-limit}
\end{corollary}

\begin{proof}
  Let $k$ be a positive integer. If $n \ge e_k$ then $a^n$ is divisible by $2^{e_k}$ but
  not divisible by 10, so $c_{10}(a^n) \ge k$ by the previous theorem.
  Therefore, $c_{10}(a^n) \ge k$ for all $n \ge e_k$. 
  Since $k$ is arbitrary, we conclude that 
  $\lim\limits_{n\to\infty} c_{10}(a^n) = \infty$.
\end{proof}

\section{Generalizing to other bases}

Our proofs rely only on divisibility properties and therefore extend naturally to arbitrary bases.

\begin{theorem}
Let $b \ge 2$ be an integer that is not a power of a prime, and let $p$ be a prime divisor of $b$.
Let $\{e_k\}$ be a sequence of positive integers such that $e_1 = 1$ and $p^{e_{k+1}} > b^{e_k}$ 
for all $k \ge 1$.
If $n$ is a positive integer that is divisible by $p^{e_k}$ but not divisible by $b$, 
then $c_{b}(n) \ge k$.
\label{general-powers-lower-bound}
\end{theorem}

\begin{proof}
  The proof is by induction on $k$. The case $k = 1$ is trivial, so let us assume that $k \ge 2$.
  By the division algorithm, there exist integers $q \ge 0$ and $0 \le r < b^{e_{k-1}}$
  such that
  $$ n = b^{e_{k-1}} q + r.$$
  
  Since $n \ge p^{e_k} > b^{e_{k-1}}$, it follows that $q \ge 1$.
  
  Since $n$ and $b^{e_{k-1}}$ are divisible by $p^{e_{k-1}}$, 
  $r$ is also divisible by $p^{e_{k-1}}$.
  Moreover, $r$ is not divisible by $b$, so 
  $c_{b}(r) \ge k - 1 $ by the induction hypothesis.
  
  Therefore,
  $$
  c_{b}(n)  = c_{b}(q) + c_{b}(r) 
          \ge 1 + (k - 1) 
          = k.
  $$
  \end{proof}


\begin{theorem}
Let $b \ge 2$ be an integer that is not a power of a prime, 
let $p$ and $q$ be distinct prime divisors of $b$,
and let $\{e_k\}$ be defined as in Theorem 2.
If $$\phi_{p,q}(n) := \nu_p(n) - \nu_q(n) \frac{\nu_p(b)}{\nu_q(b)} \ge e_k$$
then $c_p(n) \ge k$.
\label{phi-lower-bound}
\end{theorem} 


\begin{proof}
Write $n$ as $b^r m$, where $m$ is not divisible by $b$. 
The function $\phi_{p,q}$ satisfies $\phi_{p,q}(b) = 0$ and 
$\phi_{p,q}(uv) = \phi_{p,q}(u) + \phi_{p,q}(v)$ for all positive integers $u$ and $v$,
so $\phi_{p,q}(n) = \phi_{p,q}(m)$.
Since $\nu_p(m) \ge \phi_{p,q}(m) \ge e_k$ and $b \nmid m$, 
Theorem \ref{general-powers-lower-bound} implies that $c_b(m) \ge k$.
But $c_b(m) = c_b(n)$, since $m$ and $n$ have the same digits in base $b$, apart from trailing zeros.
Therefore, $c_b(n) \ge k$.
\end{proof}

\begin{corollary}
  Let $a \ge 2$ and $b \ge 2$ be integers. Suppose that $b$ has prime divisors $p$ and $q$ such that
  $\phi_{p,q}(a) > 0$.
  Then $\lim\limits_{n\to\infty} c_b(a^n) = \infty$.
\end{corollary}

\begin{proof}
Let $k > 0$ be given.
There exists an integer $N$ such that $\phi_{p,q}(a^n) = n \phi_{p,q}(a) \ge e_k$ for all $n \ge N$, so
Theorem \ref{phi-lower-bound} implies that $c_b(a^n) \ge k$ for all $n \ge N$. 
Since $k$ is arbitrary, we conclude that $\lim\limits_{n\to\infty} c_b(a^n) = \infty$.
\end{proof}

Note that this argument also shows that $c_b(a^n)$ grows at least as fast as $\Omega(\log n)$
when the conditions of the corollary are satisfied.
In 1973, Senge and Straus\cite{senge-straus1973} 
proved that if $a \ge 1$ and $b \ge 2$ are positive integers then 
$\lim\limits_{n \to \infty} c_b(a^n) = \infty$ if and only if
$\log(a) / \log(b)$ is irrational.
However, they did not prove a lower bound on the rate of growth.
In 1980, Stewart\cite{stewart1980} proved that
$$c_b(a^n) > \frac{\log \log n}{\log \log \log n + C} - 1$$
for $n > 25$, where $C$ depends on $a$ and $b$ alone.

\section{Related sequences}

Theorem 3 can be applied to many other sequences. We will give two examples here.
As before, let $b \ge 2$ be an integer that is not a prime power, 
let $p$ and $q$ be distinct prime divisors of $b$,
and let $$\phi_{p,q}(n) = \nu_p(n) - \nu_q(n) \frac{\nu_p(b)}{\nu_q(b)}.$$
%Let $b$, $p$, $q$, and $\phi_{p,q}$ be as in the previous section.

\subsection*{Factorials}

By Legendre's formula\cite[p. 263]{dickson1919}, 
$\nu_p(n!) = (n - s_p(n)) / (p - 1)$.
Thus,
$$\phi_{p, q} (n!) \approx n \left( \frac1{p-1} - \frac {\nu_p(b)} {(q-1)\nu_q(b)} \right)$$
whenever $(p-1) \nu_p(b) \ne (q-1) \nu_q(b)$, since $s_p(n) + s_q(n) \ll n$.
Therefore, $\lim\limits_{n\to\infty} c_b(n!) = \infty$ 
for any $b$ with distinct factors $p$ and $q$
satisfying 
$$(p-1) \nu_p(b) > (q-1) \nu_q(b).$$ 
In particular, $\lim\limits_{n\to\infty} c_{10}(n!) = \infty$.

\subsection*{Cumulative LCMs}

Let $\Lambda_n = \mathrm{lcm}(1, 2, \ldots, n)$.
It is easy to see that $\nu_p(\Lambda_n) = \lfloor \log_p(n) \rfloor$.
Thus,
$$\phi_{p,q}(\Lambda_n) = \lfloor \log_p n\rfloor - \lfloor \log_q n\rfloor \cdot \frac{\nu_p(b)}{\nu_q(b)}
\approx \log n \cdot \left( \frac1{\log p} - \frac1{\log q} \cdot \frac{\nu_p(b)}{\nu_q(b)}\right).$$

The quantity in parentheses is nonzero, since $\log(p)/\log(q)$ is irrational, and we may assume that it is
positive, else we can switch $p$ and $q$. Therefore, 
$\lim\limits_{n\to\infty} \phi_{p,q}(\Lambda_n) = \infty$, hence
$\lim\limits_{n \to \infty} c_b(\Lambda_n) = \infty$ for every base $b$ that is not a prime power.

{\bf Remark.} Sanna\cite{sanna2015} proved that $s_b(n!)$ and $s_b(\Lambda_n)$ are greater than
$C_b \log n \log \log \log n$ for each integer $n > e^e$ and every $b \ge 2$, where $C_b$ is a 
constant depending only on $b$.
\medskip

\printbibliography
\end{document}
