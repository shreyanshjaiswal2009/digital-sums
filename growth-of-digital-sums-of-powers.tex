% !TEX TS-program = pdflatex
% !TEX encoding = UTF-8 Unicode
\documentclass[11pt]{amsart} 

\usepackage[utf8]{inputenc}
\usepackage{geometry}
\geometry{letterpaper}
\usepackage{graphicx}
% \usepackage[parfill]{
% 	parskip
% }

%%% PACKAGES
\usepackage{amsmath}
\usepackage{amssymb}
\usepackage{sistyle}
\SIthousandsep{,}
\usepackage{tcolorbox}
\usepackage{hyperref}
\usepackage{biblatex}
\usepackage{xurl}

%%% Theorem-like environments
\newtheorem{theorem}{Theorem}
\newtheorem{corollary}{Corollary}

%%% Macros
\DeclareMathOperator*{\lcm}{\mathrm{lcm}}

\addbibresource{growth-of-digital-sums-of-powers.bib} %Import the bibliography file



\title{Elementary Bounds on Digital Sums of Powers, Factorials, and LCMs}
\author{David G Radcliffe}
\date{\today}

\begin{document}

	\begin{abstract}
		We prove that the sum of the base-$b$ digits of $a^{n}$ grows at least logarithmically in $n$, 
    provided that $\log_b(d) \not\in \mathbb{Q}$, where $d$ is the smallest factor of $a$ 
    such that $\gcd(a/d, b) = 1$.
    Our approach uses only elementary number theory and applies to a wide class of sequences, 
    including factorials and $\Lambda(n) = \lcm(1, 2, \ldots, n)$.
	\end{abstract}

	\maketitle

	\section{Introduction}

	This article was inspired by the following problem, which was posed and solved
	by Wac{\l}aw Sierpi{\'n}ski\cite[Problem 209]{sierpinski1970}:
	\begin{quote}
		\emph{Prove that the sum of digits of the number $2^{n}$ (in decimal system)
		increases to infinity with $n$.}
	\end{quote}

	We will prove a sufficient condition on positive integers $a$ and $b$ which implies
	that the sum of the base-$b$ digits of $a^{n}$ grows at least logarithmically in
	$n$. This condition includes Sierpiński's problem as a special case. The asymptotic 
  behavior of digital sums of powers has been studied by Senge and Straus (1973) and Stewart
	(1980), but their results rely on methods from transcendence theory. In contrast,
	the arguments here are elementary.

	Consider the sequence of powers of 2 (sequence \href{https://oeis.org/A000079}{A000079} 
  in the OEIS):

	\[
		1, 2, 4, 8, 16, 32, 64, 128, 256, 512, 1024, \ldots .
	\]

	This sequence grows very rapidly. Now, let us define another sequence, by adding
	the decimal digits of each power of 2. For example, 16 becomes $1+6=7$, and 32
	becomes $3+2=5$. The first few terms of this new sequence 
  (\href{https://oeis.org/A001370}{A001370}) are listed below.

	\[
		1, 2, 4, 8, 7, 5, 10, 11, 13, 8, 7, \ldots .
	\]

	Our new sequence grows much more slowly, and it is not monotonic. 
  Nevertheless, it is reasonable to conjecture that it diverges to infinity. 
  Indeed, one might guess that the sum of the decimal digits of $2^{n}$ 
  is asymptotic to $4.5 n \log_{10}2$, 
  since $2^{n}$ has $\lfloor n \log_{10}2\rfloor + 1$ decimal digits, 
  and the digits seem to be approximately uniformly distributed among $0, 1, 2, \ldots, 9$. 
  However, this stronger conjecture remains to be proved.

	For an integer $b \ge 2$, we write $s_{b}(n)$ for the sum of the base-$b$
	digits of $n$, and $c_{b}(n)$ for the number of nonzero digits in that expansion.
	These functions are asymptotic to each other, since $c_{b}(n) \le s_{b}(n) \le
	(b-1) c_{b}(n)$ for all $n$ and $b$; so we will restrict our attention to
	$c_{b}(n)$. For a prime $p$, $\nu_{p}(n)$ denotes the exponent of $p$ in the
	prime factorization of $n$.

  \section*{Digital sums of powers of two}

	Let us prove that $\lim_{n\to\infty}c_{10}(2^{n}) = \infty$. 

  Let $n$ be a positive integer, and write the decimal expansion of $2^n$ as
  $2^n = \sum_{i=0}^{\infty} d_i 10^i$, where $d_i \in \{0, \ldots, 9\}$
  and $d_i = 0$ for all but finitely many terms. 
  Note that the final digit $d_0$ of $2^{n}$ cannot be zero,
  since $2^n$ is not divisible by 10.

	If $2^{n} > 10$, then $2^{n}$ is divisible by 16; so
	$2^{n} \bmod 10^{4}$, the number formed by the last four digits of $2^{n}$, is
	also divisible by 16. If the first three of these digits were zero, then $2^{n}
	\bmod 10^{4}$ would be less than 10, which is a contradiction. So at least one
  of the digits $d_1, d_2, d_3$ is nonzero.

	If $2^{n} > 10^4$, then $2^{n}$ is divisible by $2^{14}$; so
	$2^{n} \bmod 10^{14}$, the number formed by the last 14 digits of $2^{n}$,
  is also divisible by $2^{14}$. If the first 10 of these digits were zero,
  then $2^{n} \bmod 10^{14}$ would be less than $10^{4}$, which is a contradiction.
  So at least one of the digits $d_4, d_5, \ldots, d_{13}$ is nonzero.

	We can continue in this way, finding longer and longer non-overlapping blocks
	of digits, each containing at least one nonzero digit, as illustrated in Figure 1.
  This shows that as $n$
	increases to infinity, the number of nonzero digits of $2^{n}$ also increases to
	infinity.
  
  \begin{figure}[h]
    \begin{centering}
      % $2^{50}$ = 11\framebox{2589990684}\framebox{262}\framebox{4}
      $2^{103}$ = \framebox{101412048018258352}\framebox{1197362564}\framebox{300}\framebox{8}
    \end{centering}
  \caption{Each block contains at least one nonzero digit.}
  \end{figure}

  Let us formalize this argument.

	\begin{theorem}
		Let $\{m_{k}\}$ be a sequence of integers such that 
    $m_{1} \ge 1$ and $2^{m_{k}} > 10^{m_{k-1}}$ for all $k \ge 2$. 
    If $n$ is a positive integer that is divisible by $2^{m_k}$ 
    but not divisible by 10, then $c_{10}(n) \ge k$.
		\label{base-ten-lower-bound}
	\end{theorem}

	\begin{proof}
		The proof is by induction on $k$. The case $k = 1$ is trivial, so let us assume
		that $k \ge 2$. By the division algorithm, there exist integers $q \ge 0$ and
		$0 \le r < 10^{m_{k-1}}$ such that

		\[
			n = 10^{m_{k-1}}q + r.
		\]

		Since $n \ge 2^{m_k} > 10^{m_{k-1}}$, it follows that $q \ge 1$.

		Since $n$ and $10^{m_{k-1}}$ are divisible by $2^{m_{k-1}}$, 
    $r$ is also divisible by $2^{m_{k-1}}$. 
    Moreover, $r$ is not divisible by 10,
    so $c_{10}(r) \ge k - 1$ by the induction hypothesis.

		Note that $c_{10}(n) = c_{10}(q) + c_{10}(r)$, since the digit expansion of $n$
		is the concatenation of the digit expansions of $q$ and $r$, possibly with
		leading zeros. Therefore,
		\[
			c_{10}(n) = c_{10}(q) + c_{10}(r) \ge 1 + (k - 1) = k.
		\]
	\end{proof}

  \begin{corollary}
		Let $a$ be a positive integer that is divisible by 2 but not divisible by 10.
		Then $c_{10}(a^n) \ge \log_4(n)$ for all $n \ge 1$.
    In particular,
    $\lim_{n\to \infty}c_{10}(a^{n}) = \infty$.
		\label{even-powers-base-ten-limit}
	\end{corollary}

  \begin{proof}
    Let $m_k = 4^{k-1}$ for $k \ge 1$. 
    This sequence satisfies $m_{1} \ge 1$
    and $2^{m_k} > 10^{m_{k-1}}$ for all $k \ge 2$.
    Let $n\ge 4$ be a positive integer, and let $k = \lceil \log_4 n \rceil$,
    so that $4^{k-1} < n \le 4^k$. Then $a^n$ is divisible by $2^n$,
    which is greater than $2^{m_k}$. Therefore,
    $c_{10}(a^n) \ge k \ge \log_4(n)$ by Theorem \ref{base-ten-lower-bound}.
  \end{proof}


	\section{Generalizing to other bases}

	Our proofs rely only on divisibility properties and therefore extend naturally
	to other bases.

	\begin{theorem}
		Let $b \ge 2$ be an integer that is not a power of a prime, and let $p$ be a
		prime divisor of $b$. Let $\{m_{k}\}$ be a sequence of integers such
		that $m_{1} \ge 1$ and $p^{m_{k}} > b^{m_{k-1}}$ for all $k \ge 2$. If $n$ is a
		positive integer that is divisible by $p^{m_k}$ but not divisible by $b$, then
		$c_{b}(n) \ge k$. \label{general-powers-lower-bound}
	\end{theorem}

	\begin{proof}
		The proof is by induction on $k$. The case $k = 1$ is trivial, so let us assume
		that $k \ge 2$. By the division algorithm, there exist integers $q \ge 0$ and
		$0 \le r < b^{m_{k-1}}$ such that
		\[
			n = b^{m_{k-1}}q + r.
		\]

		Since $n \ge p^{m_k}> b^{m_{k-1}}$, it follows that $q \ge 1$.

		Since $n$ and $b^{m_{k-1}}$ are divisible by $p^{m_{k-1}}$, $r$ is also
		divisible by $p^{m_{k-1}}$. Moreover, $r$ is not divisible by $b$, so $c_{b}(
		r) \ge k - 1$ by the induction hypothesis.

		Therefore,
		\[
			c_{b}(n) = c_{b}(q) + c_{b}(r) \ge 1 + (k - 1) = k.
		\]
	\end{proof}

  {\bf Notation.} Given an integer $b \ge 2$ with distinct prime divisors $p$ and $q$,
  define a function $\phi_{b,p,q}$ as follows:
	\[
		\phi_{b,p,q}(n) = \nu_{p}(n) - \nu_{q}(n) \frac{\nu_{p}(b)}{\nu_{q}(b)}.
	\]
  It is easy to verify that $\phi_{b,p,q} (b^r t) = \phi_{b,p,q}(t)$ for all $r \ge 0$
  and $t \ge 1$. Intuitively, $\phi_{b,p,q}$ is a modification of $\nu_p$
  that ignores trailing zeros in the base-$b$ representation of
  its argument.

	\begin{theorem}
		Let $b \ge 2$ be an integer that is not a power of a prime, let $p$ and $q$
		be distinct prime divisors of $b$, and let $\{m_{k}\}$ be defined as in Theorem
		\ref{general-powers-lower-bound}.
		If $\phi_{b,p,q}(n) \ge m_k$, then $c_{b}(n) \ge k$. \label{phi-lower-bound}
	\end{theorem}

	\begin{proof}
		Write $n$ as $b^{r} t$, where $t$ is not divisible by $b$. 
		Observe that $\phi_{b,p,q}(n) = \phi_{b,p,q}(t)$. 
    Since $\nu_{p}(t) \ge \phi_{b,p,q}(t) \ge m_{k}$
		and $b \nmid t$, Theorem \ref{general-powers-lower-bound} implies that
		$c_{b}(t) \ge k$. But $c_{b}(t) = c_{b}(n)$, since $t$ and $n$ have the same
		digits in base $b$, apart from trailing zeros. Therefore, $c_{b}(n) \ge k$.
	\end{proof}

	\begin{theorem}
		Let $a \ge 2$ and $b \ge 2$ be integers. Let $d$ be the smallest factor of $a$
    such that $\gcd(a/d, b) = 1$.
    In other words, $d = \prod_p p^{\nu_p(a)}$, the product being taken over
    the set of all prime divisors of $b$.
    If $\log(d) / \log(b)$ is irrational, then 
    $c_{b}(a^{n}) > \log_r n$ for all $n \ge 1$, where $r \ge 2$ is an integer
    depending only on $a$ and $b$.
    In particular, $\lim_{n\to\infty}c_{b}(a^{n}) = \infty$.
    \label{general-digit-sum}
	\end{theorem}

	\begin{proof}
    Let $p_1^{e_1} \cdots p_r^{e_m}$ be the prime factorization of $b$.
    Then $d = p_1^{f_1} \cdots p_r^{f_m}$, where some of the $f_i$ may be zero.
    If 
    \[
      \frac{f_1}{e_1} = \cdots = \frac{f_m}{e_m}
    \]
    then $\log(d) / \log(b) = f_1/e_1$, which is rational.
%
    Therefore, if $\log(d) / \log(b)$ is irrational, then the ratios
    $f_i/e_i$ are not all equal, which implies that $b$ has two prime
    factors $p = p_i$ and $q = p_j$ such that $f_i / e_i > f_j / e_j$, 
    hence $\phi_{b,p,q}(a) > 0$.
  
    Let $r = \lceil \log(b) / \log(p) \rceil$, and let $m_k = r^{k-1}$ for $k \ge 1$.
    This sequence satisfies $m_1 \ge 1$ and $p^{m_k} > b^{m_{k-1}}$
    for all $k \ge 2$.
%
    Let $n$ be a positive integer, and let $k = \lfloor \log_r n \rfloor$,
    so that $r^{k-1} < n \le r^k$. Then $a^n$ is divisible by $p^n$,
    which is greater than $b^{m_k}$. Therefore,
    $c_{b}(a^n) \ge k \ge \log_r(n)$ by Theorem \ref{phi-lower-bound}.
	\end{proof}

  In 1973, Senge and Straus\cite[Theorem 3]{senge-straus1973}
	proved that if $a \ge 1$ and $b \ge 2$ are positive integers, then
	$\lim_{n \to \infty}c_{b}(a^{n}) = \infty$ if and only if
	$\log(a) / \log(b)$ is irrational. However, they did not demonstrate a lower bound. 
  In 1980, Stewart\cite[Theorem 2]{stewart1980} proved that
	if $\log(a)/\log(b)$ is irrational, then
	\[
		c_{b}(a^{n}) > \frac{\log n}{\log \log n + C} - 1
	\]
	for $n > 4$, where $C$ depends on $a$ and $b$ alone.
  Theorem \ref{general-digit-sum} achieves a stronger bound, 
  but with stricter conditions on $a$ and $b$.

	\section{Related sequences}

	Theorem \ref{phi-lower-bound} can be applied to many other sequences. 
  We will give two examples here. 
  As before, let $b \ge 2$ be an integer, and let $p$
	and $q$ be distinct prime divisors of $b$.

	\subsection*{Factorials}

	By Legendre's formula\cite[p. 263]{dickson1919},
	$\nu_{p}(n!) = (n - s_{p}(n)) / (p - 1)$. Thus,
	\[
		\phi_{b,p, q}(n!) \approx n \left( \frac{1}{p-1}- \frac{\nu_{p}(b)}{(q-1)\nu_{q}(b)}
		\right)
	\]
	whenever $(p-1) \nu_{p}(b) \ne (q-1) \nu_{q}(b)$, since $s_{p}(n) + s_{q}(n) \ll
	n$. Therefore, $\lim_{n\to\infty}c_{b}(n!) = \infty$ for any $b$ with distinct
	factors $p$ and $q$ satisfying
	\[
		(p-1) \nu_{p}(b) > (q-1) \nu_{q}(b).
	\]
	In particular, $\lim_{n\to\infty}c_{10}(n!) = \infty$.

	\subsection*{Cumulative LCMs}

	Let $\Lambda_{n} = \lcm(1, 2, \ldots, n)$.
  This is sequence \href{https://oeis.org/A003418}{A003418} in the OEIS. 
  It is easy to see that $\nu_{p}(\Lambda_{n}) = \lfloor \log_{p}(n) \rfloor$. Thus,
	\[
		\phi_{b,p,q}(\Lambda_{n}) = \lfloor \log_{p} n\rfloor - \lfloor \log_{q} n\rfloor
		\cdot \frac{\nu_{p}(b)}{\nu_{q}(b)}\approx \log n \cdot \left( \frac{1}{\log
		p}- \frac{1}{\log q}\cdot \frac{\nu_{p}(b)}{\nu_{q}(b)}\right).
	\]

	The quantity in parentheses is nonzero, since $\log(p)/\log(q)$ is irrational,
	and we may assume that it is positive, else we can switch $p$ and $q$. Therefore,
	$\lim_{n\to\infty}\phi_{b,p,q}(\Lambda_{n}) = \infty$, hence $\lim_{n
	\to \infty}c_{b}(\Lambda_{n}) = \infty$ for every base $b$ that is not a prime
	power.

	{\bf Remark.} Sanna\cite{sanna2015} proved that $s_{b}(n!)$ and
	$s_{b}(\Lambda_{n})$ are greater than $C \log n \log \log \log n$ for every
	integer $n > e^{e}$ and every $b \ge 2$, where $C$ is a constant depending only
	on $b$.

	\medskip

	\printbibliography
\end{document}